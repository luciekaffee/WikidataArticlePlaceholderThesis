\chapter{Conclusion and outlook}

The main goal of the thesis is to increase access to free and open knowledge. Due to the implementation of the ArticlePlaceholder, it is now possible to generate content pages on the Wikipedia for Wikidata data. \\
This is not only a theoretical improvement but also wished by the communities as visible in the discussion pages on the local Wikipedias. (See Appendix) \\
In the future, many of the aspects mentioned before will be implemented such as the possibility of content translation, starting the article creation with data from Wikidata, adjusting the extension also for sister projects, and making the ArticlePlaceholder pages indexable. \\
\\
To evaluate how useful the extension is for the different Wikipedias, it will be necessary to observe how many meaningful articles were created and whether they make the need of bot created stubs superfluous. \\
In further research it would be possible to look into natural language processing and creating text from the data displayed. Since this will not necessarily be something the Wikipedia community wishes, it would still be helpful for other use cases of MediaWiki. 