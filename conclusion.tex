\chapter{Conclusion and outlook}

The main goal of this thesis was to increase access to free and open knowledge by providing access to information in languages that are often not considered. Thanks to the implementation of the ArticlePlaceholder, it is now possible to generate content pages on Wikipedia for Wikidata's free and open data. \\
This is not only a theoretical improvement but also desired by the communities as evident in the input pages on the local Wikipedias. \\
\\
To evaluate how useful the extension will be for the different Wikipedias, it will be necessary to observe how many meaningful articles will be created with its help and whether they will render bot created stubs superfluous. \\
Possible further research could look into natural language processing and creating text from data. \\
\\
The development of the ArticlePlaceholder is ongoing. \\
In the future, many of the features mentioned in this thesis, but not yet realized will be certainly implemented such as the option of content translation, starting the article creation with data from Wikidata, adjusting the extension for sister projects, and making the ArticlePlaceholder pages indexable. The extension offers a basis to realize this and already implements many of the features that may change the accessibility of under-resourced languages for the better. 
