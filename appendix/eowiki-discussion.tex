\begin{quote}
 Anstataŭigilo de neekzistantaj artikoloj

En la retpoŝta konferenco de Vikidatumoj aperis propono (angle) por malgrandaj Vikipedioj por ŝalti aldonaĵon ArticlePlaceholder (fakte anstataŭigilo de neekzistantaj paĝoj). Ĝi funkcias tiel ke kiam uzanto entajpas ion en serĉilon, la serĉilo ne proponas nur artikolojn el Vikipedio, sed ankaŭ erojn en Vikidatumoj (tiuj kun titolo aŭ kromnomo en la koncerna lingvo) kaj post iro al tiu "paĝo", ĝi montros datumojn el Vikidatumoj rekte en Vikipedio kune kun butono por krei novan paĝon. Ofte tiel oni povas trovi almenaŭ bazajn datumojn pri personoj (datoj kaj lokoj de naskiĝo kaj morto, ks.) eĉ se neniu ankoraŭ verkis artikolon pri la homo en koncerna lingvo. \\
\\
Ĉar Vikidatumoj enhavas pli ol 675 000 erojn en Esperanto dum kiam E-Vikipedio havas nur pli ol 220 000 artikolojn, tio povus utili ankaŭ por ni. La kreinto de la aldonaĵo ŝajnas ke ĝi interesiĝas pri daŭra evoluo kaj plibonigo de la aldonaĵo. Pli da informoj estas en la konferenco. Mi nur ne scias kion oni konsideras kiel "malgranda Vikipedio", sed eble tio ne estus problemo. --Venca24 (diskuto) 18:47, 20 Jan. 2016 (UTC) \\
\\
    Mi subtenas la elprovon de tiu iniciato kaj nian engaĝiĝon en ĝi. Esence temas pri integro de io kiel Reasonator rekte en Vikipedion, kaj tio estas sencohava kaj multe pli trovebla kaj uzanto-amika ol la "Serĉorezultoj de Vikidatumoj" kiun ni jam nun redonas kiel simplan liston tute ĉe la fino de la paĝo kun serĉrezultoj kiam ne ekzistas aparta artikolo pri temo. La antaŭvidebla efiko por nia Vikipedio estas granda, do el tiu vidpunkto ni certe eniras la difinon de la serĉata "malgranda Vikipedio". Apero de Esperanto en studentlaboraĵo kaj nova projekto ligita al la Vikipedia programaro cetere iĝos bona reklamo por nia lingvo. Problemo povus esti nur se iu ero apenaŭ enhavas utilajn informojn, la paĝo estas preskaŭ malplena kaj tio ĝenos la legantojn – sed ni vidu, ĉu tio efektive ĝenus al iu. Ja preskaŭ ĉiu ero enhavos minimume titolon, priskribon kaj bildon, kio jam igas la paĝon ne aspekti tute malplena. Se aldone estos klare el la paĝoj, ke tio estas nur aŭtomate generitaj informoj, kaj la indeco de kreo de nova artikolo restos same okulfrapa kiel nun, por ne fortimigi redaktuntojn, tio estos kontentiga. Sola mia problemo estas, ke mi persone dum la nunaj monatoj ne povos tion asisti, do necesos ke iu alia anonciĝu al ŝi nianome kaj aranĝu la aferon (tamen la teksto diras, ke ne nepre necesos granda kunlaboro, se ni ne sufiĉe spertos aŭ homkapablos). Venca24, ĉu vi povus la kontakton kun ŝi flegi? Cetere, la aŭtoro ja estas ŝi kaj mi ne rimarkis, ke ŝi intencus iel kaŝi tion – male, la elekto de Ada Lovelace kiel specimeno estas bela atentigo pri tiu fakto kaj laŭ mi jam pro la bezono de plia engaĝiĝo de virinoj en komputiko tiu ŝia projekto estas subteninda kaj ŝia sekso neprisilentinda. --Marek "Blahma" BLAHUŠ (diskuto) 09:11, 21 Jan. 2016 (UTC) \\
\\
        Mi skribis al ĝi ke ni interesiĝas kaj kion necesas fari por enkonduki la ilon en nian Vikipedion. \\
        Mi ne konas la homon kaj nenio en la mesaĝo gvidis min al ideo ke ĝi estas certe ŝi. Se ĝi estus ĉeĥo, tiam estus tute klare ke ĝi estas ŝi laŭ nomo, la nomon Aimée mi ne konas kaj mi ne volis esplori ĉu ie en la mondo ne ekzistas Lucie kiel vira nomo, pro tio mi elektis neŭtralan manieron. Same mi ne scias ĉu ĝi certe estas germano kaj tial mi ne menciis tion (tio ankaŭ ne kaŝindas), sed neniun ĝenas tio ĉi. Mi samopinias ke indas havi inojn inter informadikistoj kaj ke oni ne kaŝu tion, sed ĉi tie ne temas pri prezento de la persono, sed pri prezento de la ilo. Tio, ke ĝi elektis Ada ankaŭ ne certigis min, ĉar kiam mi pripensis prelegi dum UK pri inoj en informadiko, Ada estis mia unua elekto (ĉe ŝi multo komencis), sed tio ne signifas ke mi estas ino. --Venca24 (diskuto) 12:34, 21 Jan. 2016 (UTC) \\
\\
            Necesas traduki la ilon en Translatewiki (mi ne sukcesis ankoraŭ ekhavi permeson por traduki) kaj havi subtenon de la komunumo - do mi kreas baloton por simple esprimi (mal)subtenon por ke ankaŭ neparolantoj de Esperanto povu simple kompreni la rezulton. Diskuto povas daŭri ĉi tie. --Venca24 (diskuto) 14:00, 21 Jan. 2016 (UTC) \\
\\
                Mi jam tradukis ĉion en E-on. Kiu havas aliron al TranslateWiki, ĝi povas kontroli (aŭ plibonigi) la tradukojn. --Venca24 (diskuto) 07:30, 22 Jan. 2016 (UTC)

                    Mi subtenas la elprovon de tiu iniciato. Mi ne certas, ke la ilo estus vere utila por kreantoj sed ni bezonas vikidatumojn kaj iniciato sciigas pri vikidatumoj. Vikidatumoj faciligi la kreanton de novaj artikoloj sed ĉefe ĝi estas necesa por niaj ĝidatigoj. Komuneco pri datumoj estas nia intereso, por personoj ni havas vere informoj en vikidatumoj sed por geografiaj lokoj estas malfacila, la grandaj vikipedioj havas iliaj proprajn metadatumoj kaj ili ne komunikas gravegajn informojn kiel loĝantaro. Se oni konas kaj uzas vikidatumojn, oni povos argumenti en nia lingvo vikipedio por peti komuniki liajn informojn. --pino (diskuto) 19:21, 22 Jan. 2016 (UTC)

Mi subtenas la testadon ĉe ni! Ĉi-semjnfine (supozeble kun Uzanto:LaPingvino kaj eble aliaj esperantistoj) mi partoprenos FOSSDEM (evento por programistoj en Bruselo, Belgio), kie prelegos ankaŭ la kreanto de la programo kaj do mi persone babilos kun ŝi kaj rekomendos nian Vikipedion. Ĉu eble iu aliĝas al ni? --KuboF (diskuto) 19:23, 29 Jan. 2016 (UTC)

    Do, mi persone renkontiĝis kun la programisto kaj mi konfirmis, ke nia Vikipedio interesiĝas pri ŝia etendaĵo. Ŝi observas nian Diskutejon kaj ni iom parolis pri ŝia subteno al Esperanto kaj Ido (kaj iom miris pri la komento de Dominik :)). Do rezulte, nia Vikipedio havas bonan ŝancon por servi kiel frua testanto, sed pro studado kaj alia laboro de la programisto, necesos atendi ĉirkaŭ 1-2 monatoj. --KuboF (diskuto) 21:19, 6 Feb. 2016 (UTC)

Baloto

Ĉu vi subtenas ŝalton de ilo ArticlePlaceholder en Esperanto-Vikipedio?
Por

    Por Por: --Venca24 (diskuto) 14:00, 21 Jan. 2016 (UTC)
    Por Por: --Polyglot (diskuto) 17:56, 21 Jan. 2016 (UTC)
    Por Por: --pino (diskuto) 19:21, 22 Jan. 2016 (UTC)
    Por Por: --Haruo (diskuto) 21:26, 24 Jan. 2016 (UTC)
    Por Por: --KuboF (diskuto) 19:16, 29 Jan. 2016 (UTC)

Kontraŭ

    Kontraŭ Kontraŭ: ŝajnas ke tiu uzanto (http://www.mediawiki.org/wiki/User:Frimelle) jam pli favoras al Ido ol Esperanto. Li/Ŝi ne bezonas nin. --Dominik (diskuto) 14:27, 21 Jan. 2016 (UTC)

    De kie vi prenis tiun ideon pri favoro al Ido? --Venca24 (diskuto) 16:59, 21 Jan. 2016 (UTC)

        @Venca24: - Lia/Ŝia retejo estas inventaire.io --Dominik (diskuto) 18:16, 21 Jan. 2016 (UTC)

            Mi ne scias ĉu vi pensas tion serioze aŭ ne. La vorto Inventaire estas ŝajne franclingva, la retejo estas ĝenerale anglalingva kaj .io estas domajno de Brita Hindoceana Teritorio kaj la libroj ĉe mi nenion havas komunan kun Ido. Diru al mi kion mi misvidis. --Venca24 (diskuto) 19:09, 21 Jan. 2016 (UTC)
            Eble mia enkonduka frazo povas soni strange - pli bone mi dirus: Mi ne scias ĉu vi ŝercas aŭ ĉu mi estas blinda. (Ne hezitu respondi pri tio. ;-) ) --Venca24 (diskuto) 07:30, 22 Jan. 2016 (UTC)

                Ĉu Brita Hindoceana Teritorio estas la sola lando en la mondo kie Ido oficialiĝis? (mi ŝercas) --Haruo (diskuto) 17:26, 24 Jan. 2016 (UTC)

                    @Haruo: Ŝajne jes kaj tio mojosas. Bonvolu baloti. :-) --Venca24 (diskuto) 19:36, 24 Jan. 2016 (UTC)

Detenas

    Sindetena Sindetena: lasu lin labori en granda vikio antaŭ provi ŝin en malgranda vikio. --Kabhi2011 (diskuto) 15:42, 21 Jan. 2016 (UTC)

\end{quote}
