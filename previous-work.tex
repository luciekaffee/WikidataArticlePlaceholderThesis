\chapter{Previous work}
\section{Infobox}
\begin{quote}
	Roughly every third Wikipedia article contains an infobox --- a table that displays important facts about the subject in attribute-value form
\end{quote}
found \citet[5]{infobox}. Infoboxes are fixed-format tables on (in case of left to right languages) the top right of Wikipedia pages which display a summary on the data of an article's topic. \citep{wiki:05} They are static tables created by editors via templates. They can display data from Wikidata but as of today mostly do not. They are of great value for the articles and allow the reader to get a quick overview on an issue. \\
It is important to consider how infoboxes work but still differ the layout of the Article Placeholder so the two are not confused with each other.

\section{Reasonator}
Reasonator \footnote{\url{https://tools.wmflabs.org/reasonator/}} by Magnus Manske is written mostly in JavaScript. It has a very similar approach to the Article Placeholder extension. It displays items from Wikidata in a visually appealing way, aimed at readers rather than Wikidata editors. Reasonator does not stick to the typical Wikipedia layout but tries to find the best way possible to display the data. \\
The items are when appropriate separated in the classes  "people, locations, [and] species" according to \citet{wiki:06}. Therefore it is possible to display the data in a way that is context sensitive. \\

\section{Lsjbot}
In some cases, the number of articles on a Wikipedia is in no relation to the number of editors or speakers of the Wikipedia's language. This is especially interesting when looking at Swedish- a language with only 10.5 million speakers worldwide as \citet{nlpd:01} states. The Swedish Wikipedia is the second largest Wikipedia in concern of articles. As of 15.01.2016 the vast majority (56.3\%) of these articles are about taxons. \\
In the Cebuano Wikipedia this is even more obvious. It has the third most articles of all Wikipedias. \\
Cebuano is the spoken in the Philippines. Cebuano speakers make up 24\% of the population in the country. \citep{cebuano:01} They are the second largest ethnolinguistic group of the Philippines with 18.5 million people. \citep{cebuano:02} \\
As of 17.01.2015 95\% of all articles on Cebuano Wikipedia are marked in Wikidata as an \textit{``instance of'' (P31)} \textit{``taxon'' (Q16521)}. \citep{wiki:07} \\
Many of these pages were created by \textit{bots}. A bot in the context of Wikipedia is ``an automated or semiautomated tool that carries out repetitive or mundane tasks'' \citep{wiki:08} written by editors of Wikipedia. \\
In the two examples before the bot in charge for most of these pages on taxons is the \textit{Lsjbot}\footnote{\url{https://sv.wikipedia.org/wiki/Wikipedia:Projekt_DotNetWikiBot_Framework/Lsjbot/Makespecies}}. As of July 2014 it has created 2.7 million articles, ``two thirds of which appear in the Cebuano Wikipedia [...]; the other third appear in the Swedish Wikipedia'' \citep{wiki:09}. Those articles are usually \textit{stub articles} --- ``article[s] considered too short to give an adequate introduction to a subject'', explains \citet{stubs}. \\
Even though it helps providing much more information than these small languages would usually have access to or their editors are able to create, the issue remains that most of the time the communities are not able to maintain these articles by themselves. It's not possible for the limited number of editors to watch all the created stub articles and keep the data of these up to date. \\
Looking at the Lsjbot the goal was to make the data maintainable in one central place (Wikidata) so editors from different communities can contribute to the data displayed and help keeping it up to date. Additionally, not creating an article but just displaying data has the advantage of being able to encourage the editors to actually create an article with more information that can be maintained by them.

\section{Wdsearch.js}
Wdsearch.js \todo{the link}\footnote{\url{https://en.wikipedia.org/w/index.php?title=MediaWiki:Wdsearch.js&action=raw&ctype=text/javascript}} is a script that includes search results from Wikidata on the Wikipedia search page. It is written in JavaScript and enabled per default on, for example, English Wikipedia and Italian Wikipedia states \citet{gerardm:01, gerardm:02}.
The idea was aimed to work on the problem, that according to \citet{manske:01} ``over 60\% of Wikidata items have no corresponding article in the English Wikipedia; once we leave the “top five”, this exceeds 90\%.'' \\
Using this script, search results from Wikidata are added to the bottom of the normal Wikipedia search result page. The label of the item contains a link to the Wikidata item. A short description gives the user an overview, if it is the topic they were looking for. The user can choose to read the topic on either a different Wikipedia, see multimedia content on Wikimedia Commons, or see data on Resonator, using three buttons. \\
To date, the script is enabled on Polish and Italian Wikipedia. It helps users find more information on a topic they are looking for but as of now, the user is forced to visit another website. Knowledge of how to navigate and use this other website is required. \\
The idea to include Wikidata search results is something Article Placeholder picks up and tries to improve. The Wikidata search results are supposed to be added to the Wikipedia search page, too, but the user should stay on their language Wikipedia.