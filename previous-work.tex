\section{Previous work}
\subsection{Infobox}
\todo{connect to wiki:05}
Infoboxes are tables on the top right of Wikipedia pages which display a summary of an articles topic. They are static tables created by editors. They can display data from Wikidata but up to date mostly don't. They are the first and most important way data is displayed on Wikipedia. They are of great value for the articles and allow the reader to get a quick overview. \\
In the work with Article Placeholder it is very important to consider the way infoboxes work but to differ in the overall layout strongly to emphasize the fact that the tables created by the extension are not infoboxes but auto-generated content. 

\subsection{Reasonator}
Reasonator \footnote{https://tools.wmflabs.org/reasonator/} by Magnus Manske written mostly in JavaScript. It has a very similar approach to the Article Placeholder extension- it displays Items from Wikidata in a visually appealing way, aiming at readers rather than Wikidata editors. Reasonator does not stick with the typical Wikipedia layout but tries to find the best way possible to display the data. \\
The items are- when appropriate- separated in the classes  \todo{connect to wiki:06} "people, locations, [and] species". Therefore it is possible to display the data in a way that responds to the subject matter. \\

\subsection{Wdsearch.js}
There is a script in JavaScript\todo{the link}\footnote{\href{https://en.wikipedia.org/w/index.php?title=MediaWiki:Wdsearch.js&action=raw&ctype=text/javascript}{Script on English Wikipedia}}, which is on enabled on the Italian and other Wikipedias, to include search results from Wikidata on Wikipedia. At the bottom of the regular search page on Wikipedia it displays results from Wikidata. The label of the item contains a link to the Wikidata item. It is followed by some important values taken from the properties on an item followed by the description if one is available. With three icons following a user can choose to see the topic on a different Wikipedia, Wikimedia Commons or Reasonator. \\
To date, the script is enabled on Polish and Italian Wikipedia. It helps users to find more information on a topic they are looking for but forces them, for now, to go to another website in order to find this information and understand the way these other websites work. \\
The idea to include Wikidata search results is something Article Placeholder picks up and tries to improve. The Wikidata search results are supposed to be added to the Wikipedia search page, too, but the user should stay on their language Wikipedia. Instead of properties describing an item only the description if available in the language should be displayed. 

\subsection{ruwiki (connecting to Wikidata)}