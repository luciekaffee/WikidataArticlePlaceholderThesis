\chapter{Previous work}
\section{Infobox}
\todo{connect to wiki:05}
Infoboxes are fixed-format tables on the top right of left to right language Wikipedia pages which display a summary on the data of an article's topic. They are static tables created by editors. They can display data from Wikidata but up to date mostly do not. They are the first and most important way data is displayed on Wikipedia. They are of great value for the articles and allow the reader to get a quick overview on an issue. \\
It is important to consider how infoboxes work but still differ the layout of the Article Placeholder so the two are not confused with each other.

\section{Reasonator}
Reasonator \footnote{https://tools.wmflabs.org/reasonator/} by Magnus Manske is written mostly in JavaScript. It has a very similar approach to the Article Placeholder extension. It displays items from Wikidata in a visually appealing way, aimed at readers rather than Wikidata editors. Reasonator does not stick to the typical Wikipedia layout but tries to find the best way possible to display the data. \\
The items are when appropriate separated in the classes  \todo{connect to wiki:06} "people, locations, [and] species". Therefore it is possible to display the data in a way that adapts to the subject matter. \\

\section{Lsjbot}
In some cases, the number of articles on a Wikipedia is in no relation to the number of editors or speakers of the Wikipedia's language. This is especially interesting when looking at Swedish- a language with only 10.5 million speakers worldwide.\footnote{\href{http://www.npld.eu/about-us/swedish/}{Reference}} The Swedish Wikipedia is the second biggest Wikipedia in concern of articles. As of 15.01.2016 the vast majority (56.3\%) of these articles are about taxons. \\
In the Cebuano Wikipedia this is even more clear. It has the third most articles of all Wikipedias. \\
Cebuano is the spoken in the Philippines. Cebuano speakers make up 24\% of the population in the country. \footnote{\href{https://terpconnect.umd.edu/~oard/pdf/hlt03.pdf}{Resource}} They are the second biggest ethnolinguistic group of the Philippines with 18.5 million people. \footnote{\href{http://www.britannica.com/topic/Cebuano-language}{Reference}} \\
As of 17.01.2015 95\% of all articles on Cebuano Wikipedia are of type taxon.\footnote{\href{https://www.wikidata.org/wiki/Wikidata:Statistics/Wikipedia}{Reference}} \\
Many of these edits are performed by \textit{bots}. A bot in the context of Wikipedia is ``an automated ot semiautomated tool that carries out repetitive or mundane tasks'' \footnote{\href{https://en.wikipedia.org/w/index.php?title=Wikipedia:Bots&oldid=662582073}{reference}} written by editors of Wikipedia. \\
In the two examples before the bot in charge for most of these pages is the \textit{Lsjbot}\footnote{\href{https://sv.wikipedia.org/wiki/Wikipedia:Projekt_DotNetWikiBot_Framework/Lsjbot/Makespecies}{Source code of Lsjbot}}. As of July 2014 it has created 2.7 million articles, ``two thirds of which appear in the Cebuano Wikipedia [...]; the other third appear in the Swedish Wikipedia''\footnote{\href{https://en.wikipedia.org/wiki/Lsjbot}{Reference}}. Those articles are usually \textit{stub articles} -- articles with usually one or two sentences and an image. \\
Even though it helps providing much more information than these small languages would usually have access to or their editors are able to create, the issue is that most of the time the communities are not able to maintain these articles by themselves. It's not possible for the limited number of editors to watch all the created articles and keep the data up to date. \\
Looking at this project the goal was to make the data maintainable in one central place (Wikidata) so editors from different communities can contribute to the data display and help keeping it up to date. Additionally, not creating an article but just displaying data has the advantage of being able to encourage the editors more to actually create an article with more information that can be maintained by them. 

\section{Wdsearch.js}
Wdsearch.js \todo{the link}\footnote{\href{https://en.wikipedia.org/w/index.php?title=MediaWiki:Wdsearch.js&action=raw&ctype=text/javascript}{Script on English Wikipedia}} is a script by \todo{author} that includes search results from Wikidata on the Wikipedia search page. It is written in JavaScript and enabled on e.g. Italian and Polish Wikipedia. \\
Using this script, search results from Wikidata are added to the bottom of the normal Wikipedia search result page. The label of the item \todo{kennt der Leser noch nicht} contains a link to the Wikidata item. A short description gives the user an overview, if it is the topic they were looking for. The user can choose to read the topic on either a different Wikipedia, see multimedia content on Wikimedia Commons or Resonator, using three buttons. \\
To date, the script is enabled on Polish and Italian Wikipedia. It helps users find more information on a topic they are looking for but as of now, the user is forced to visit another website. Knowledge of how to navigate and use this other website is required. \\
The idea to include Wikidata search results is something Article Placeholder picks up and tries to improve. The Wikidata search results are supposed to be added to the Wikipedia search page, too, but the user should stay on their language Wikipedia.

\section{ruwiki (connecting to Wikidata)}