\section{Previous work}
\subsection{Infobox}
\todo{connect to wiki:05}
Infoboxes are fixed-format tables on the top right of left to right language Wikipedia pages which display a summary on the data of an article's topic. They are static tables created by editors. They can display data from Wikidata but up to date mostly do not. They are the first and most important way data is displayed on Wikipedia. They are of great value for the articles and allow the reader to get a quick overview on an issue. \\
It is important to consider how infoboxes work but still differ the layout of the Article Placeholder so the two are not confused with each other.

\subsection{Reasonator}
Reasonator \footnote{https://tools.wmflabs.org/reasonator/} by Magnus Manske is written mostly in JavaScript. It has a very similar approach to the Article Placeholder extension. It displays items from Wikidata in a visually appealing way, aimed at readers rather than Wikidata editors. Reasonator does not stick to the typical Wikipedia layout but tries to find the best way possible to display the data. \\
The items are when appropriate separated in the classes  \todo{connect to wiki:06} "people, locations, [and] species". Therefore it is possible to display the data in a way that adapts to the subject matter. \\

\subsection{Wdsearch.js}
Wdsearch.js \todo{the link}\footnote{\href{https://en.wikipedia.org/w/index.php?title=MediaWiki:Wdsearch.js&action=raw&ctype=text/javascript}{Script on English Wikipedia}} is a script by \todo{author} that includes search results from Wikidata on the Wikipedia search page. It is written in JavaScript and enabled on e.g. Italian and Polish Wikipedia. \\
Using this script, search results from Wikidata are added to the bottom of the normal Wikipedia search result page. The label of the item \todo{kennt der Leser noch nicht} contains a link to the Wikidata item. A short description gives the user an overview, if it is the topic they were looking for. The user can choose to read the topic on either a different Wikipedia, see multimedia content on Wikimedia Commons or Resonator, using three buttons. \\
To date, the script is enabled on Polish and Italian Wikipedia. It helps users find more information on a topic they are looking for but as of now, the user is forced to visit another website. Knowledge of how to navigate and use this other website is required. \\
The idea to include Wikidata search results is something Article Placeholder picks up and tries to improve. The Wikidata search results are supposed to be added to the Wikipedia search page, too, but the user should stay on their language Wikipedia.

\subsection{ruwiki (connecting to Wikidata)}