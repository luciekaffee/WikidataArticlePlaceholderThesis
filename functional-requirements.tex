\documentclass[11pt]{article}

\usepackage{hyperref}

\title {{Functional requirements}}
\author {Lucie-Aim\'{e}e Kaffee}
\date{}

\begin {document}

\section {Functional requirements}
The functional requirements are based on the non-functional requirements and aim to implement those.
There are also requirements, that are based on the circumstances and the techniques used. \\
To be able to use the data of Wikidata, ArticlePlaceholder needs to base on Wikibase and mediawiki.
As mediawiki is implemented in PHP, the main part of the extension needs to make use of the language as well as the services provided by mediawiki and Wikibase. \\
The language for scripts on the Wikipedia is Lua. Therefore the parts that will be user-editable need to be written in Lua so the display can be overwritten by the local communities. \\
The default for the content pages should present the information and data in a useful way and make use of the images in Wikimedia Commons, that are connected to the item displayed.\\
The content pages generated by the extension will need to work on mobile. \\
\\
As described before, the requirements would differ from the different stages in the deployment cycle. They are always built on top of each other but since the parts have different importance we would focus on a very basic set up first and add up more features from there. 

\subsection{Test system}


\begin{itemize}
\item A default display of the data
\item Show main image of the item separated from the statement list
\item Get to the ArticlePlaceholder via the SpecialPage URL

\item Localisation of the whole extension
\item Show the article connected to the item, if an article on the wiki exists
\item Get to the Article Placeholders via search
\item A possibility to create an article from scratch from an Article Placeholder by giving the user the possibility to enter a title

\item Create an Article from the ArticlePlaceholder
\end{itemize} 
 

\end {document}