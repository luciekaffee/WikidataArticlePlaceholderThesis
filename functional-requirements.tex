\documentclass[11pt]{article}

\usepackage{hyperref}

\title {{Functional requirements}}
\author {Lucie-Aim\'{e}e Kaffee}
\date{}

\begin {document}

\section {Functional requirements}
The functional requirements are based on the non-functional requirements and aim to implement them.
There are also requirements, that are based on the circumstances and the techniques used. 
As mediawiki is implemented in PHP, the main part of the extension needs to make use of the language as well as the services provided by mediawiki and Wikibase. To be able to use the data of Wikidata, ArticlePlaceholder needs to base on Wikibase and mediawiki. \\
The language for scripts on the Wikipedia is Lua. Therefore the parts that will be user-editable need to be written in Lua so the display can be overwritten by the local communities. \\
The default for the content pages should present the information and data in a useful way and make use of the images in Wikimedia Commons, that are connected to the item displayed.\\
The content pages generated by the extension will need to work on mobile. \\
 

\end {document}