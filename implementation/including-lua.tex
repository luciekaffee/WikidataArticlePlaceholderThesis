\section{Including PHP in Lua}\label{including-lua}

In order to pass the reurn values of PHP functions to Lua, functionality provided by Scribunto was used. \\
The functions accessible by the Lua code are declared in the class \texttt{\justify Scribunto\char`_LuaArticlePlaceholderLibrary}. This class extends \texttt{\justify Scribunto\char`_LuaLibraryBase} in order to register the Lua interface. \\
To get the image property set in \texttt{\justify LocalSettings.php}, the globale variable \texttt{\justify \$wgArticlePlaceholderImageProperty} is wrapped in an array and returned in the function \texttt{\justify getImageProperty()}. Lua functions can have multiple return values \citep{luabook:01}, thus Scribunto needs an array in order to model that. \\
In Lua, these PHP callbacks are copied from \texttt{\justify mw\char`_interface} to the local variable \texttt{php}. From this variable the PHP functions in \texttt{\justify Scribunto\char`_LuaArticlePlaceholderLibrary} can be invoked. Therefore the image property ID can now be used in Lua as well by invoking \texttt{\justify getImageProperty()} in \texttt{\justify mw.ext.articleplaceholder.entityRenderer}. \\
Getting the array with the ordered properties requires a service, the \texttt{ArticlePlaceholderService}.It returns an instance of iteself. It's function \texttt{\justify getPropertyOrderProvider()} returns an instance of the \texttt{\justify PropertyOrderProvider} described in Chapter~\ref{ordering-stat}: \nameref{ordering-stat}. This follows the \textit{Singelton pattern}, which enforces, according to \citet{designpattern}, ``that only one instance of a class will be created''.
In \texttt{\justify Scribunto\char`_LuaArticlePlaceholderLibrary} in the function \texttt{\justify Scribunto\char`_LuaArticlePlaceholderLibrary} this service is called and the associative array of ordered properties is returned.\\
Similar to the image property, the function can be invoked in the Lua module.
