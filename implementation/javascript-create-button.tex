\subsection{Create article button (JavaScript}
		
For the layout it was necessary to have a button at the top of the page asking for the creation of the article. \\
The JavaScript module to create this button uses \textit{OOjs UI}, which is included in MediaWiki and ``is a library that allows developers to rapidly create front-end web applications that operate consistently across a multitude of browsers.'' \citep{wiki:27} \\
A dialog appears and asks the user for a title for the new article. The default title refers to the label of the item, which can be adjusted. \\
There should be an option for users who have JavaScript disabled. But since the number of users with JavaScript disabled is about 3\% according to \citet{wiki:02} \todo[color=red!40]{wo gelten diese 3\%? } the priority is rather low and is not considered for implementation now. \\
\\
The button is added to the page in the PHP class \texttt{\justify SpecialAboutTopic} via the function \texttt{\justify showCreateArticle( \$label )}. The label of the ArticlePlaceholder is passed in order to display it in the dialog window that is created in the JavaScript module \texttt{\justify ext.articleplaceholder.createArticle}. \\
In this module, an event listener is added to the button so it will open the dialog window on click, the \texttt{CreateArticleDialog}. This dialog has an input field, displaying the label of the item. This string can be changed by the user. In the dialog is an additional submit button, which is an instance of the class \texttt{OO.ui.ButtonWidget}. In order to redirect the user to an edit page when the submit button is clicked, an event listener is added to the button. Before redirecting there is a check to see if the page that is supposed to be edited already exists. If that's not the case, the user is redirected with the help of \texttt{\justify document.location.href}. \\
In the case of the page already existing, an error message is added to the dialog window at the bottom, and the user is asked to enter a new title.