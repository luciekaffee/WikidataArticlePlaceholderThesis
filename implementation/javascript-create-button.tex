\subsection{Create article button (JavaScript}
		
Using the existing JavaScript modules as provided by MediaWiki, there is a button on the top of the page asking for the creation of the article. A pop-up at the page opens and asks the user for a title for the new article. The default is the label of the item, which can be adjusted. \\
There should be an option for users with disabled JavaScript. But since the amount of users with JavaScript disabled is about 3\% according to \citet{wiki:02} \todo[color=red!40]{wo gelten diese 3\%? } the priority is rather low and is not considered for now. \\
\\
The button is added to the page generated by the SpecialPage in the PHP class \texttt{SpecialAboutTopic} in the function \texttt{showCreateArticle( \$label )}. The label of the ArticlePlaceholder is passed in order to display it in the pop-up window, that is created in the JavaScript module \texttt{ext.articleplaceholder.createArticle}. \\
In this module, an event listener is added to the button so it will open the pop-up window on click, the \texttt{CreateArticleDialog}. This dialog has an input field, displaying the passed Sting containing the label of the item. This String can be edited or removed by a user. In the dialog is an addtional submit button, which is an instance of the class \texttt{OO.ui.ButtonWidget}. In order to redirect the user to an empty page when the submit button is clicked, an event listener is added to the button. Before redirecting it is checked whether the page that is supposed to be edited already has content. If that's not the case, with the help of \texttt{document.location.href} the user is redirected. \\
In the case of the page already existing, an error message is added to the dialog window at the bottom. 