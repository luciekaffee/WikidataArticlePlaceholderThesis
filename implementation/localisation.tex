\subsection{Localisation}

The extension includes an i18n folder in order to be able to localize all strings needed by the extension. Every message needed by the extension has a unique message key. The mapping of the key value pairs are stored in JSON. \\
The keys need to be unique not only in the ArticlePlaceholder extension but also must not conflict with MediaWiki and its other extensions, the keys start with \textbf{articleplaceholder-}. Due to MediaWiki conventions, they are all lower case and may not contain spaces. \todo{link: mediawiki Developing extensions -- Localisation} \\
The documentation for every message is stored in the qqq.json file. The other messages are in the json files with the appropriate language key. While developing it's only necessary to define the English and documentation messages since the other messages are translated by the community on \href{https://translatewiki.net/}{translatewiki.net}. \\
To load the messages, their directory is passed to \href{https://www.mediawiki.org/wiki/Manual:$wgMessagesDirs}{\textbf{\$wgMessagesDirs}}. \textbf{\$wgMessagesDirs} is an associative array, that maps the extension name to message directory for the MediaWiki software to extract the messages and their keys. \\
The messages can now be used in PHP, JavaScript and Lua with their respective methods provided by MediaWiki. In PHP the message can be loaded using the function \textbf{wfMessage} with the message key as parameter. The syntax for Lua is \textbf{mw.message.new( \textit{'message-key'} ):plain()} and for JavaScript \textbf{mw.message( \textit{'message-key'} ).escaped()}. Due to the similar syntax of Lua and JavaScript being rather similar, the call to the function is similar, too. In the last two examples, the messages are additionally escaped.