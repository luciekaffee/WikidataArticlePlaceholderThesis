\section{Internationalization}

The extension includes an \texttt{\justify i18n} folder to be able to localize all strings needed by the extension. This is the way predefined by MediaWiki \citep{wiki:34}.  Every message needed by the extension has a unique message key. These keys are mapped to the text in the different languages. The key-value pairs are stored as JSON. \\
The keys need to be unique not only in the ArticlePlaceholder extension but also must not conflict with MediaWiki's and its other extensions' keys. Therefore they start with \texttt{\justify articleplaceholder-}. Due to MediaWiki conventions, they are all lower case and may not contain spaces. \\
The documentation for every message is stored in the \texttt{\justify qqq.json} file. The other messages are in the JSON files with the appropriate language key. While developing, it is only necessary to define the English and documentation messages, since the other messages are translated by the community on \url{translatewiki.net}.
\begin{quote}
 \citet{wiki:26} states ``\url{translatewiki.net} is a web-based translation platform, powered by the Translate extension for MediaWiki [...] [It has] 6000 translators for over 50[,000] pages from over [twenty] projects including MediaWiki, OpenStreetMap, Mifos, Encyclopedia of Life and MantisBT.''
\end{quote}

To load the messages, their directory is added to \texttt{\justify \$wgMessagesDirs}. \texttt{\justify \$wgMessagesDirs} is an associative array that maps the extension name to the appropriate message directory for the MediaWiki software to extract the messages and their keys. \\
The messages can now be used in PHP, JavaScript and Lua with their respective methods provided by MediaWiki. In PHP the message can be loaded using the function \texttt{\justify wfMessage} with the message key as a parameter. The syntax for Lua is \texttt{\justify mw.message.new( message-key ):plain()} and for JavaScript \texttt{\justify mw.message( message-key ).escaped()}. Due to the similar syntax of Lua and JavaScript, the function call is similar, too. \\ 
In the last two examples, the messages are additionally escaped.