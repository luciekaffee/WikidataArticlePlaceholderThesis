\section{Deployment}

To actually release the extension, a new component to MediaWiki, certain steps in cooperation with Wikimedia need to be done. 
The release itself needs, according to \citet{deployment}, 
\begin{quote}
	release, installation, activation, deactivation, update, and removal of components. These activities constitute a large and complex process that we refer to as software deployment.  
\end{quote}

In Wikimedia, people with appropriate rights, volunteers and staff, can deploy. They have a schedule, which can be adjusted of a extension author's needs. Due to the author of the thesis to date does not have deployment rights, cooperation with Wikimedia staff was necessary in order to do the steps described in figure~\ref{fig:deployTimeline}. \\
\begin{sidewaysfigure}
	\todo[inline]{look into this more!}
	\begin{tikzpicture}[snake=zigzag, line before snake = 5mm, line after snake = 5mm]
		% draw horizontal line   
		\draw (0,0) -- (2,0);
		\draw[snake] (2,0) -- (4,0);
		\draw (4,0) -- (5,0);
		\draw[snake] (5,0) -- (7,0);
		\draw (7,0) -- (8,0);
		\draw[snake] (8,0) -- (10,0);
		\draw (10,0) -- (12,0);
		\draw[snake] (12,0) -- (14,0);
		\draw (14,0) -- (16,0);
		\draw[snake] (16,0) -- (18,0);
		\draw (18,0) -- (19.5,0);
		

		% draw vertical lines
		\foreach \x in {0,1.5,4.5,7.5,10.5,12,13.5,15,16.5,18,19.5}
			\draw (\x cm,3pt) -- (\x cm,-3pt);

		% draw nodes
		\draw (0,0) node[below=3pt] {} node[above=3pt] {$   $};
		\draw (1.5,0) node[below=3pt] {\small test system} node[above=3pt] {};
		\draw (3,0) node[below=3pt] {} node[above=3pt] {\small security review};
		\draw (4.5,0) node[below=3pt] {\small beta Wikipedia} node[above=3pt] {};
		\draw (6,0) node[below=3pt] {} node[above=3pt] {\small feedback};
		\draw (7.5,0) node[below=3pt] {\small test Wikipedia} node[above=3pt] {};
		\draw (9,0) node[below=3pt] {} node[above=3pt] {\small feedback};
		\draw (10.5,0) node[below=3pt] {\small beta feature 1} node[above=3pt] {};
		\draw (12,0) node[below=3pt] {} node[above=3pt] {\small feedback};
		\draw (13.5,0) node[below=3pt] {\small beta feature 2} node[above=3pt] {};
		\draw (15,0) node[below=3pt] {} node[above=3pt] {\small feedback};
		\draw (16.5,0) node[below=3pt] {\small enable} node[above=3pt] {};
		\draw (18,0) node[below=3pt] {} node[above=3pt] {\small more Wikipedias};
		\draw (19.5,0) node[below=3pt] {\small sister projects} node[above=3pt] {};
	\end{tikzpicture}
    \caption{Time line of deployments}
    \label{fig:deployTimeline}
\end{sidewaysfigure}

\paragraph{Test system}
From the beginning there was a test setup available on Wikimedia Labs\footnote{\url{articleplaceholder.wmflabs.org/mediawiki}}. Wikimedia Labs is 
\begin{quotation}
	the Wikimedia Foundation (WMF)'s cloud computing environment for developing software for the Foundation's operations. It also hosts bots and tools maintained and used by the community to maintain the foundation's projects. 
\end{quotation} \citep{wiki:03}
Besides the bots and scripts running on Wikimedia Labs it is used for various tools and to test software by volunteers as well as WMF's staff.\todo{improve sentence} \\
The test setup was established to give editors as well as the public a first insight into what the extension looks like and what features it offers and to make them a part of development. It is also aimed at starting a discussion on what can be improved in order to be adjusted to the community's needs.

\paragraph{Security review}
To deploy an extension to the Wikimedia cluster, it is necessary to go through the process of the security review by the Security Engineer of the Wikimedia Foundation. \todo{in order to?}
  
\paragraph{Beta Cluster}
The Beta Cluster\footnote{\url{http://en.wikipedia.beta.wmflabs.org/wiki/Main_Page}} is the \textit{WMF integration environment} and used for testing. In order to deploy the extension there, someone with the appropriate deployment rights needed to do code review and then deploy the extension. First issues with a real Wikipedia cluster were identified and adjusted in this step. \\
Parallel to this step, the communities of all Wikipedias were asked, whether any of them would be interested in being the first Wikipedia deploying the extension. Wikipedias interested were asked to open discussion pages to gather the support of the community. (For e-mail send to the \textit{Wikipedia Technical Ambassadors} mailing list, and an extract of the community survey pages see Appendix \ref{})

\paragraph{Test Wikipedia}
Test Wikipedia\footnote{\url{https://test.wikipedia.org/wiki/Main_Page}} is another test system by Wikimedia. It is connected to Test Wikidata\footnote{\url{https://test.wikidata.org/wiki/Wikidata:Main_Page}}. From a development perspective, this is the last step of deployment before enabling the extension on a real Wikipedia. This is also another chance to gain more attention and collect input before the Beta feature. Issues can be found and fixed before they break things on a Wikipedia, that is used by a community. \\
\\
These deployment were done in the scope of the thesis. The following deployments will need more adjustment and therefore are not part of the thesis but will follow in the near future. 

\paragraph{Beta Feature}
Beta Features are functionalities that can be enabled by registered user in their preferences. 
\begin{quotation}
	The primary purpose of Beta Features is to allow for Wikimedia designers and engineers (from the Wikimedia Foundation and community alike) to roll out technical improvements in an environment where large numbers of users can test, give feedback, and use these features in real-world settings.
\end{quotation} \citep{wiki:04}

The next step was to deploy the extension as a beta feature. While having certain requirements for the test set up, the beta feature was supposed to be actually used by the community and therefore needed to fulfill more requirements, building up on what was already archived in the step before. \\
After the first deploy as a beta feature, the feedback by the community needs to be evaluated and the extension needs to be adjusted accordingly. The adjusted beta feature needs not only to consider the feedback again but also to gain the support and trust of the community to be enabled on a Wikipedia by default.

\paragraph{Deploy to a Wikipedia, enable by default}
To have it actually deployed, it must match more requirements. Ideally it would be deployed on a Wikipedia, that has a limited amount of articles and editors first, to actually see if it fulfills the needs of the readers it is aiming at and accepted by the existing communities. 

\paragraph{Deploy to more Wikipedias}
It would be then necessary and possible to deploy it to more Wikipedias, that want to participate. 

\paragraph{Deploy to sister projects}
In the future, it would be great to adjust the extension in a way that it's useful to other Wikimedia projects as well. This is out of the scope for the thesis but something to keep in mind while developing. \\
One of the possible next projects could be Wikivoyage due to its structural similarity to Wikipedia.