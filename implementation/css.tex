\subsection{Style elements (CSS)}

In order to style the layout elements, CSS classes are assigned in the Lua module. Their looks as well as the create article button's look are adjusted in \texttt{ext.articleplaceholder.defaultDisplay.css}. \\
So as not to conflict with other MediaWiki style elements, the CSS classes are prefixed with \texttt{\justify articleplaceholder-}. \\
The \texttt{\justify divs} are nested so that, for example, the main image and the identifier (which both have their own \texttt{div}) are in one \texttt{div} named after its place in the layout \texttt{\justify articleplaceholder-sidebar}. Additionally the identifiers are all in one \texttt{div}, the \texttt{\justify articleplaceholder-identifierlist}. \\
The \texttt{divs} containing the statement groups have a maximum width, so that a maximum of three boxes are in one row. The number of columns differs depending on the number of statements. When the browser window is smaller, the number of boxes per row adjusts accordingly. Initially it was planned to adjust them to a tiling layout, but since tiling layout in pure CSS would expect the boxes to be ordered vertically in columns this was not possible. \todo{[citation needed]} It is important to show the most important information first, otherwise the ordering of statement groups would be pointless. \\
In order to be responsive, the extension makes use of \textit{media queries}. Media queries are a convenient way to add CSS styles for elements that need to be adjusted for different devices. \citep[43]{mediaquery}\\
In MediaWiki with the \textit{resource loader} \texttt{\justify \$wgResourceModules} media queries can be assigned to CSS modules. This way it is possible to load another CSS module (\texttt{\justify ext.articleplaceholder.defaultDisplaySmall.css}), when the screen size is smaller then 930 pixels. This was added in order to avoid the overlapping of the sidebar with the image and identifiers, and the statement groups.