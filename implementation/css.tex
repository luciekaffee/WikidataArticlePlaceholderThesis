\subsection{Style elemts (CSS)}

In order to style the layout elements, CSS classes were assigned in the Lua module. Their looks as well as the create article button's look are adjusted in \texttt{ext.articleplaceholder.defaultDisplay.css}. The main elements belonging in one part in the layout such as the main image and the identifier, that are in one sidebar, are in one common \texttt{div}. Additionally the identifier are all in one \texttt{div}, the \texttt{sidebar}. \\
To not conflict with other MediaWiki style elements, the CSS classes are prefixed with \texttt{\justify articleplaceholder-}. \\
The \texttt{divs} containing the statement groups have a maximum width, so a maximum of three boxes are in one row. The number of columns differs depend on the amount of statements. When the browser window is smaller, the amount of boxes per row adjusts accordingly. Initially it was planned to adjust them to a tiling layout, but since tiling layout in pure CSS would expect the boxes to be ordered vertically in columns this was not possible. It is important to show the most important information in the first row, otherwise the ordering of statement groups would be pointless. \\
In order to be responsive, the extension makes use of \textit{media queries}. Media queries are a convenient way to add new CSS styles for elements that need to be adjusted for different devices. \citep[43]{mediaquery}\\
In MediaWiki with the \textit{resource loader} \texttt{\justify \$wgResourceModules} media queries can be assigned to CSS modules. This way it is possible to load another CSS module (\texttt{\justify ext.articleplaceholder.defaultDisplaySmall.css}), when the screen size is smaller then 930 pixel. This was mainly added in order to avoid the overlapping of the sidebar with image and identifier, and the statement groups.