\subsubsection{Identifier}

To get a list of all external identifier in Wikidata, the item "Wikidata property representing a unique identifier" (Q19847637), is used. All properties representing an identifier are an "instance of" (P31) this item. \\
To query all those identifier Wikidata's SPARQL endpoint \footnote{\href{https://query.wikidata.org}{https://query.wikidata.org}} is used. The following query returns a list of IDs. This list can be downloaded as a file with the data in their JSON representation. \\

\begin{lstlisting}[frame=single] 
PREFIX wd: <http://www.wikidata.org/entity/>
PREFIX wdt: <http://www.wikidata.org/prop/direct/>

SELECT ?identifier WHERE {
   ?identifier  wdt:P31 wd:Q19847637 . 
}
\end{lstlisting}
In the Lua script \texttt{Main.lua} the script \texttt{Identifier.lua} is generated from the JSON output. \\  \texttt{Identifier.lua} returns a Lua table, which contains tables with the identifier mapped to "true" to be able to check whether a property is an identifier or not. This new script is included in the Lua module and is called to get access to the table with the identifier. It is static since it would cause too many requests to the SPARQL endpoint to check for every call to the special page again if the data on identifier changed. That implies that if something changes with the identifiers on Wikidata, or a new one is added, the article placeholder doesn't reflect that. \\
This implementation was chosen due to the fact that the Wikidata team is working on a identifier data type, which would make the need for a Lua table with all identifiers superfluous. As soon as there is an identifier data type, the way of checking a property id could be adjusted to check for the data type of the respective property. 
