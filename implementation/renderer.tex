\subsection{Renderer}

Every part of an entity, or item in the case of the ArticlePlaceholder, needs a \textit{renderer} to be displayed. The renderer pulls data from Wikidata and renders it to HTML. It sets CSS classes for the HTML elements in order to style them with the CSS module \texttt{\justify ext.articleplaceholder.defaultDisplay.css}. The HTML output of the module is displayed on the \texttt{\justify Special:AboutTopic} ArticlePlaceholder. \\
The MediaWiki conventions for naming Lua modules is 
\begin{center}
\texttt{\justify mw.ext.extensionName.moduleName} 
\end{center}
In order to match them, the renderer for the ArticlePlaceholder is called 
\begin{center}
\texttt{\justify mw.ext.articlePlacholder.entityRenderer}
\end{center}
Every renderer has getter and setter functions. Therefore every part of the display can be overwritten locally on Wikipedia. \\
The Lua module on the Wikipedia is on the page \texttt{\justify Module:AboutTopic}, which is called by the template \texttt{\justify Template:AboutTopic}. In the module AboutTopic a user with the appropriate rights can call the get and set functions. \\
The renderer are as decoupled as possible so that for example the \texttt{\justify snaksRenderer} can be used by the \texttt{\justify refererenceRenderer} as well as the \texttt{\justify qualifierRenderer}. \\
The Wikibase client provides a Scribunto interface with modules to access the repository. Wherever possible, the functions provided by this interface were used. This way, the description renderer for example simply calls \texttt{\justify mw.wikibase.description} and passes the entity Id as a parameter. The \texttt{\justify descriptionRenderer} of the ArticlePlacheolder's EntityRenderer is still needed since there must be a possibility for the user to overwrite the renderer. Additionally a CSS class needs to be assigned to the result of Wikibase's description renderer in order to style it properly. \\
The EntityRenderer module itself takes just the entity Id of an item an ArticlePlaceholder is created for. The Lua table for an entity is loaded in the renderEntity function. This table represents a whole entity with all its respective data. Therefore it is performance-wise the most expensive operation in the module. \\

\begin{figure}[H]
	\centering
	\includegraphics[width=\textwidth]{diagrams/EntityRendererMethods.png}
	\caption{Diagram of all renderer functions}
	\label{fig:renderer}
\end{figure}

\todo[inline]{3 examples, one easy, two tricky, ``additionally there are renderer for ...'' and digram}