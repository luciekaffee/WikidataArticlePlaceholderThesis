\subsubsection{Renderer}

Every part of an entity, or Item in the case of the Article Placeholder, needs a renderer to be displayed. The renderer uses the data given from Wikidata and renders it to html, which can be displayed on the special page about an item. \\
The MediaWiki conventions for naming Lua module is \texttt{mw.ext.extensionName.moduleName}. In order to match them, the renderer for the ArticlePlaceholder is called \texttt{mw.ext.articlePlacholder.entityRenderer}. \\
Every renderer has get and set functions. Therefore every part of the display can be overwritten locally on Wikipedia. \\
The Lua module on the Wikipedia in on the page \texttt{Module:AboutTopic}, which is called by the template \texttt{Template:AboutTopic}. In the module AboutTopic a user with the respective rights can call the get and set functions. \\

\todo{Describe every renderer? }