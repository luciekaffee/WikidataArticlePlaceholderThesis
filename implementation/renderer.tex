\subsubsection{Renderer}

Every part of an entity, or item in the case of the ArticlePlaceholder, needs a renderer to be displayed. The renderer uses the data taken from Wikidata and renders it to HTML, which can be displayed on the special page about an item. \\
The MediaWiki conventions for naming Lua modules is 
\begin{center}
\texttt{mw.ext.extensionName.moduleName} 
\end{center}
In order to match them, the renderer for the ArticlePlaceholder is called 
\begin{center}
\texttt{mw.ext.articlePlacholder.entityRenderer}
\end{center}
Every renderer has getter and setter functions. Therefore every part of the display can be overwritten locally on Wikipedia. \\
The Lua module on the Wikipedia in on the page \texttt{Module:AboutTopic}, which is called by the template \texttt{Template:AboutTopic}. In the module AboutTopic a user with the appropriate rights can call the get and set functions. \\
\todo[inline]{3 examples, one easy, two tricky, ``additionally there are renderer for ...'' and digram}

\todo{Describe every renderer? }