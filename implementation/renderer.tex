\subsection{Renderer}

Every part of an entity---or item in the case of the ArticlePlaceholder---needs a \textit{renderer} to be displayed. The renderers are written in Lua. They pull data from Wikidata and render it to HTML, and set CSS classes for the HTML elements, in order to style them with the CSS module \texttt{\justify ext.articleplaceholder.defaultDisplay.css}. The HTML output of the module is displayed on the \texttt{\justify Special:AboutTopic} ArticlePlaceholder. \\
The MediaWiki conventions for naming Lua modules is 
\begin{center}
\texttt{\justify mw.ext.extensionName.moduleName} 
\end{center}
In order to match them, the default renderer for ArticlePlaceholders is called 
\begin{center}
\texttt{\justify mw.ext.articlePlacholder.entityRenderer}
\end{center}
For individual renderers the entity renderer has getter and setter functions. Therefore every part of the display logic can be overwritten locally on Wikipedia. \\
On the page \texttt{\justify Module:AboutTopic} on the Wikipedia is the Lua module which is invoked by \texttt{\justify Template:AboutTopic}. On the entity renderer instance obtained, a user with the appropriate rights can call the getter and setter functions. \\
The renderers are as interchangeable as possible, so that for example the \texttt{\justify snaksRenderer} can be used by the \texttt{\justify referenceRenderer} as well as the \texttt{\justify qualifierRenderer}. Therefore the scopes of the renderers are therefore limited to their functionality. \\
Wikibase provides a Scribunto interface with functions to access the repository. Wherever possible, the functions provided by this interface were used. This way, the description renderer for example simply wraps \texttt{\justify mw.wikibase.description}. The \texttt{\justify descriptionRenderer} of the ArticlePlacheolder's EntityRenderer is still needed as a separate service, since there must be a possibility for the user to overwrite the renderer. Additionally, a CSS class is wrapped around the result of Wikibase's description renderer in order to style it properly. \\
The entity ID of the item, an ArticlePlaceholder is created for, is passed to the EntityRenderer module. The Lua table for the entity is loaded in the \texttt{\justify renderEntity} function. This table represents the whole entity with all its respective data. Therefore it is a very performance-costly operation. \\

\begin{figure}[H]
	\centering
	\includegraphics[width=\textwidth]{diagrams/EntityRendererMethods.png}
	\caption{Diagram of all renderer functions}
	\label{fig:renderer}
\end{figure}