\subsubsection{SpecialPage}

The SpecialPage SpecialAboutTopic.php uses multiple services provided by either Wikibase or MediaWiki. It also extends the class SpecialPage.php provided by MediaWiki to use its functionalities. \\
To create a content page, it is necessary to first check, if there is actually an entity ID passed by the HTML form. In the case of a user arriving at the special page, the entity ID is obviously empty. Therefore the HTML of the special page would be created with the function create form, which adds HTML to the OuptputPage object and a form for the entity ID the user can enter. \\
If there is an entity ID, it needs to be checked whether this entity actually exists. This is done with the entity lookup provided by Wikibase. If the entity ID does not exist, the user will see the HTML of createForm again with an error message.
If the entity ID exists, the special page needs to check whether there is already an article on this Wiki for the item. This is done with the siteLinkLoopup service by Wikibase. The function getArticleOnWiki is called and returns the title if there is an article or null if not. The user is redirected with the redirect function of the OutputPage class to the article if it exists. \todo{this sentence is chaos!} \\
The way of the code to show a placeholder looks like the following flowchart. \\

\tikzstyle{block} = [rectangle, draw, 
    text width=5em, text centered, rounded corners, minimum height=4em]
\tikzstyle{error} = [draw, ellipse, node distance=3cm,
    minimum height=2em]
\tikzstyle{line} = [draw, -latex']
\tikzstyle{cloud} = [draw, ellipse, node distance=3cm,
    minimum height=2em]
\tikzstyle{final} = [diamond, draw, 
    text width=4.5em, text badly centered, node distance=3cm, inner sep=0pt]

\sffamily

\begin{tikzpicture}[node distance = 2cm, auto]
    % Place nodes
    \node [block] (execute) {execute};
    \node [block, below of=execute] (showContent){show content};
    \node [block, below of=showContent] (itemIdParam) {getItemId Param};
    \node [cloud, below of=itemIdParam] (entityId) {entityId?};  
    \node [block, below of=entityId] (createForm) {createForm};
    \node [final, right of=createForm] (Html) {HTML};
    \node [cloud, left of=createForm] (hasEntity) {hasEntity?};
    \node [cloud, below of=hasEntity] (onWiki) {articleOnWiki?};
    \node [error, right of=onWiki] (error) {\textcolor{red}{error}};
    \node [final, below of=onWiki] (redirect) {redirect};
    \node [final, right of=redirect] (placeholder) {show Placeholder};
    % \node [final, right of=placeholder] (JS) {JavaScript};
    % Draw edges
    \path [line] (execute) -- (showContent);
    \path [line] (showContent) -- (itemIdParam);
    \path [line] (itemIdParam) -- (entityId);
    % \path [line] (entityId) -- (createForm);
    \path [line] (entityId) -- (createForm) node [midway, above, sloped, -latex'] (textnode) {no};
    \path [line] (createForm) -- (Html);
    \path [line] (entityId) -- (hasEntity) node [midway, above, sloped, -latex'] (textnode) {yes};
    % \path [line] (entityId) -- (hasEntity);
    \path [line] (hasEntity) -- (onWiki) node [midway, above, sloped, -latex'] (textnode) {yes};
    %\path [line] (hasEntity) -- (onWiki);
    \path [line] (hasEntity) -- (error) node [midway, above, sloped, -latex'] (textnode) {no};
    % \path [line] (hasEntity) -- (error);
    \path [line] (error) -- (createForm);
    \path [line] (onWiki) -- (redirect) node [midway, above, sloped, -latex'] (textnode) {yes};
    % \path [line] (onWiki) -- (redirect);
    \path [line] (onWiki) -- (placeholder) node [midway, above, sloped, -latex'] (textnode) {no};
    % \path [line] (onWiki) -- (placeholder);
    % \path [line] (placeholder) -- (JS);
\end{tikzpicture}

\rmfamily

Otherwise an article placeholder is created. In order to do this, a template on the Wiki is invoked, which calls the Lua module. Additionally a OOUI is enable in order to include the JavaScript module. The label of the item is passed to the JavaScript module. The label is used in PHP to set the title of the page, too. The link to other Wikipedia language versions (language links) are set in PHP as well as the links to other projects. The sitelinks are read from the SiteStore service provided by Wikibase. They are assembled from their language code and the page name with a colon in between. \\
Showing a placeholder can be described as in the following chart. \\ 

\tikzstyle{block} = [rectangle, draw, 
    text width=5em, text centered, rounded corners, minimum height=4em]
\tikzstyle{line} = [draw, -latex']

\sffamily

\begin{tikzpicture}[node distance = 3cm, auto]
    % Place nodes
    \node [block, align=center] (showPl) {show Placeholder};
    \node [block, below of=showPl] (callTemp) {call template};
    \node [block, left=2em of callTemp] (showCA){showCreate Article};
	\node [block, right=2em of callTemp] (showTitle) {showTitle};
	\node [block, right=2em of showTitle] (showLl) {showLang-uagelinks};
	\node[block, below of=showCA] (passL) {pass label};
	\node [block, right=2em of passL] (enableOOUI) {enableOOUI};
	\node [block, below of=passL] (createB) {create Button};
	\node [block, right of=createB] (JS) {JavaScript module};
	\node [block, below of=showLl] (sl) {getSitelinks ForItem};
	\node [block, below of=sl] (lcpn) {Language Code:Page Name};
	
    % Draw edges
    \path [line] (showPl) -- (showCA);
	\path [line] (showPl) -- (callTemp);
	\path [line] (showPl) -- (showTitle);
	\path [line] (showPl) -- (showLl);
	\path [line] (showCA) -- (enableOOUI);
	\path [line] (showCA) -- (passL);
	\path [line] (enableOOUI) -- (createB);
	\path [line] (passL) -- (createB);
	\path [line] (createB) -- (JS);
	\path [line] (showLl) -- (sl);
	\path [line] (sl) -- (lcpn);
\end{tikzpicture}

\rmfamily