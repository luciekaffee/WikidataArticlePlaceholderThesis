\section{SpecialPage}

The SpecialPage \texttt{\justify Special:AboutTopic} uses multiple services provided by either Wikibase or MediaWiki. It also extends the class \texttt{\justify SpecialPage.php} provided by MediaWiki to use its functionalities. The class is called \texttt{\justify SpecialAboutTopic} and can be found in the namespace \texttt{\justify ArticlePlaceholder\\Specials}. \\
From a users perspective there are two ways of passing an entity ID to the extension. There is the option to do that directly via the URL, for exaple in the following way. 
\begin{center}
\colorbox{Gray}{\lstinline[basicstyle=\ttfamily\color{white}]|Special:AboutTopic/Q5279|}
\end{center}
The other option is to enter the ID on the SpecialPage. The SpecialPage is created by an HTML form in the function \texttt{\justify createForm()}, which adds HTML to the \texttt{\justify OutputPage} object provided by MediaWiki. //
When the entity ID is passed, there is a check whether this entity actually exists. This is done with the \texttt{\justify EntityLookup} provided by Wikibase. \\ If the entity ID does not exist, the user will get to the HTML output of \texttt{\justify createForm} again with an error message. \\
If the entity ID exists, the SpecialPage needs to check whether there is already an article on this Wiki for the item. This is done with the \texttt{\justify SiteLinkLookup} service by Wikibase. The function \texttt{\justify getArticleOnWiki( \$entityId)} is called. In case of an article existing it returns the title of that article. The user is then forwarded to that article with the \texttt{\justify redirect( \$url )} function of the \texttt{\justify OutputPage} class.
The way of the code to show a placeholder is shown in Figure~\ref{fig:createpl}. 
\begin{figure}[H]
	\centering
	\tikzstyle{block} = [rectangle, draw, 
    text width=5em, text centered, rounded corners, minimum height=4em]
\tikzstyle{error} = [draw, ellipse, node distance=3cm,
    minimum height=2em]
\tikzstyle{line} = [draw, -latex']
\tikzstyle{cloud} = [draw, ellipse, node distance=3cm,
    minimum height=2em]
\tikzstyle{final} = [diamond, draw, 
    text width=4.5em, text badly centered, node distance=3cm, inner sep=0pt]

\sffamily

\begin{tikzpicture}[node distance = 2cm, auto]
    % Place nodes
    \node [block] (execute) {execute};
    \node [block, below of=execute] (showContent){show content};
    \node [block, below of=showContent] (itemIdParam) {getItemId Param};
    \node [cloud, below of=itemIdParam] (entityId) {entityId?};  
    \node [block, below of=entityId] (createForm) {createForm};
    \node [final, right of=createForm] (Html) {HTML};
    \node [cloud, left of=createForm] (hasEntity) {hasEntity?};
    \node [cloud, below of=hasEntity] (onWiki) {articleOnWiki?};
    \node [error, right of=onWiki] (error) {\textcolor{red}{error}};
    \node [final, below of=onWiki] (redirect) {redirect};
    \node [final, right of=redirect] (placeholder) {show Placeholder};
    % \node [final, right of=placeholder] (JS) {JavaScript};
    % Draw edges
    \path [line] (execute) -- (showContent);
    \path [line] (showContent) -- (itemIdParam);
    \path [line] (itemIdParam) -- (entityId);
    % \path [line] (entityId) -- (createForm);
    \path [line] (entityId) -- (createForm) node [midway, above, sloped, -latex'] (textnode) {no};
    \path [line] (createForm) -- (Html);
    \path [line] (entityId) -- (hasEntity) node [midway, above, sloped, -latex'] (textnode) {yes};
    % \path [line] (entityId) -- (hasEntity);
    \path [line] (hasEntity) -- (onWiki) node [midway, above, sloped, -latex'] (textnode) {yes};
    %\path [line] (hasEntity) -- (onWiki);
    \path [line] (hasEntity) -- (error) node [midway, above, sloped, -latex'] (textnode) {no};
    % \path [line] (hasEntity) -- (error);
    \path [line] (error) -- (createForm);
    \path [line] (onWiki) -- (redirect) node [midway, above, sloped, -latex'] (textnode) {yes};
    % \path [line] (onWiki) -- (redirect);
    \path [line] (onWiki) -- (placeholder) node [midway, above, sloped, -latex'] (textnode) {no};
    % \path [line] (onWiki) -- (placeholder);
    % \path [line] (placeholder) -- (JS);
\end{tikzpicture}

\rmfamily
	\caption{Flowchart for creating a placeholder}
	\label{fig:createpl}
\end{figure}

If there is no article for the passed item, an ArticlePlaceholder is created. In order to do this, a template on the Wiki is invoked, which calls the Lua module. Additionally \texttt{\justify OOUI} is enable in order to include the JavaScript module. The label of the item is passed to the JavaScript module \texttt{\justify ext.articleplaceholder.createArticle}. The label is used in PHP to set the title of the page, too. The link to other Wikipedia language versions (language links) are set in PHP. The \textit{sitelinks} are read from the \texttt{\justify SiteStore} service provided by Wikibase. They are assembled from their language code and the page name with a colon in between. For example, the page linking to English Wikipedia article on ``Ada Lovelace'' would be \texttt{\justify en:Ada Lovelace}\\
Showing a placeholder can be described as in the chart in Figure~\ref{fig:showpl}. 
\begin{figure}[H]
	\centering
	\tikzstyle{block} = [rectangle, draw, 
    text width=5em, text centered, rounded corners, minimum height=4em]
\tikzstyle{line} = [draw, -latex']

\sffamily

\begin{tikzpicture}[node distance = 3cm, auto]
    % Place nodes
    \node [block, align=center] (showPl) {show Placeholder};
    \node [block, below of=showPl] (callTemp) {call template};
    \node [block, left=2em of callTemp] (showCA){showCreate Article};
	\node [block, right=2em of callTemp] (showTitle) {showTitle};
	\node [block, right=2em of showTitle] (showLl) {showLang-uagelinks};
	\node[block, below of=showCA] (passL) {pass label};
	\node [block, right=2em of passL] (enableOOUI) {enableOOUI};
	\node [block, below of=passL] (createB) {create Button};
	\node [block, right of=createB] (JS) {JavaScript module};
	\node [block, below of=showLl] (sl) {getSitelinks ForItem};
	\node [block, below of=sl] (lcpn) {Language Code:Page Name};
	
    % Draw edges
    \path [line] (showPl) -- (showCA);
	\path [line] (showPl) -- (callTemp);
	\path [line] (showPl) -- (showTitle);
	\path [line] (showPl) -- (showLl);
	\path [line] (showCA) -- (enableOOUI);
	\path [line] (showCA) -- (passL);
	\path [line] (enableOOUI) -- (createB);
	\path [line] (passL) -- (createB);
	\path [line] (createB) -- (JS);
	\path [line] (showLl) -- (sl);
	\path [line] (sl) -- (lcpn);
\end{tikzpicture}

\rmfamily
	\caption{Flowchart for showing a placeholder}
	\label{fig:showpl}
\end{figure}
