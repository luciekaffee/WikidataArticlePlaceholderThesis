\section{Smart red links}\label{sec:redLinks}
	Red links on Wikipedia display the title of an article that does not exist yet, but is desired, by linking to an empty editing page. It is difficult to connect a title to a Wikidata item since the item does not necessarily have the same label as the title of the page. Therefore connecting red links to ArticlePlaceholder is rather demanding, and not within the scope of this thesis. \\
	However, there is an ongoing discussion on how this could be realized, which is summarized in this chapter \citep{wiki:22}. \\
	\\
	The ideal plan would be to have \textit{smart red links} which are connected to a Wikidata item. In order for this to work, Wikidata sitelinks could be used. Basically there would be \textit{virtual sitelinks}, so editors could connect links to non-existing articles on Wikidata the same way that links to existing articles would be added. This way, the text of the article and its Wikitext would not change. On Wikidata, those links could either be in an own section or be displayed in red. \\
	There are multiple ways for editors to add those sitelinks to Wikidata. Existing red links would have to be migrated by editors. Clicking a red link that is not yet connected could lead the editors to a page asking them for the same article in another language and then automatically add the red link to the corresponding Wikidata item. If none exists there would still be the possibility to just add the Wikidata item ID. \\
	If editors add a red link, they could be asked to add it to the corresponding Wikidata item before saving their edit. \\
	In the case of an editor using content translation, the link would be added automatically since the article is connected to an article in another language. \\
	\\
	As this would also give editors an overview over wanted pages in Wikidata, there would be more benefits than only having ArticlePlaceholder connected to red links, such as being able to find translatable articles easily.