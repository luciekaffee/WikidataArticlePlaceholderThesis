\subsection{MediaWiki}

MediaWiki is the free and open software which is used to run Wikipedia. It is a ``free and open source wikipackage written in PHP''\footnote{\href{https://www.mediawiki.org/wiki/MediaWiki}{Reference}} using a MySQL database. Originally it was developed for Wikipedia but by now it is used by many other Wiki projects inside and outside the Wikimedia world like Wikimedia Commons and WikiLeaks. It is developed by the Wikimedia Foundation and MediaWiki volunteers. It was initially released on January 25, 2002\footnote{\href{https://en.wikipedia.org/wiki/MediaWiki}{Reference}}. Due to the extensive use of Wikipedia, MediaWiki is optimized to handle huge amounts of data and high traffic\footnote{\href{https://www.mediawiki.org/w/index.php?title=Manual:What_is_MediaWiki\%3F&oldid=743778}{Reference}}. \\
To adjust the software to the users needs, it has 878 configuration settings\footnote{\href{https://www.mediawiki.org/wiki/Manual:Configuration_settings_\%28alphabetical\%29}{Configuration Settings (alphabetically)}}. \\
MediaWiki just provides the bases for many other projects like Wikibase. To enable or change various features 2,386 extensions are currently available\footnote{\href{https://www.mediawiki.org/wiki/Category:All_extensions}{All extensions (alphabetically)}}. One of these extensions is Wikibase. \\
Pages in MediaWiki are grouped in \textit{namespaces}. Namespaces are used to sort pages according to their purpose. They are indicated in the page title with a prefix such as \texttt{Namespace:PageTitle}. Pages without a prefix are in the \textit{mainspace}. Namespaces can be localized. \footnote{\href{https://www.mediawiki.org/wiki/Help:Namespaces}{Reference}}