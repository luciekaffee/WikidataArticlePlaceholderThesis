\section{Lua and Scribunto}

Lua is an open-source scripting language. It is fast and lightweight and is used mainly in embedded systems, often to customize other software \citep{lua:01}. \\
\\
Lua is not object-oriented, but through its tables and meta tables can be easily adapted for multiple purposes. It is a dynamically typed language. According to \citet[9]{luabook:01}, it supports eight basic data types: \textit{nil}, \textit{boolean}, \textit{number}, \textit{userdata}, \textit{function}, \textit{thread}, \textit{string}, and \textit{table}. A wide variety of data structures such as \textit{arrays} can be represented with Lua's tables. \\
Lua is used with the Scribunto extension in MediaWiki. Scribunto ``[p]rovides a framework for embedding scripting languages into MediaWiki pages'' states \citet{wiki:19}. At the moment it only supports Lua. \\
\\
Scribunto allows MediaWiki users to write their own Lua modules in the \texttt{\justify Module} namespace. These can be invoked on any other wiki page (usually via \textit{Templates}) using the following command:
\begin{lstlisting}[frame=single] 
{{#invoke:module name|function name|argument}}
\end{lstlisting}

In the case of a module called \textit{Module:AboutTopic} with the function \textit{showData} and a string parameter with an entity ID, calling the module on a wiki page would look like this:
\begin{lstlisting}[frame=single] 
{{#invoke:AboutTopic|showData|Q42}}
\end{lstlisting}

The code of the Lua modules is run only when the page is parsed, which means that the modules create output only when the page is edited. For this reason, modules can't handle dynamic user input \citep{wiki:20}. Scribunto can parse Wikitext.
