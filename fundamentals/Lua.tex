\documentclass[11pt]{article}

\usepackage[utf8]{inputenc}
\usepackage[dvipsnames]{xcolor}
\usepackage{hyperref}
\usepackage{todonotes}
\usepackage{listings}
\usepackage{soul}
\usepackage{tikz}
\usetikzlibrary{shapes,arrows,snakes}
\usepackage{courier}
\usepackage{graphicx}
\usepackage{lineno}
\lstset{basicstyle=\ttfamily,breaklines=true}

\begin {document}

\subsection{Lua}

Lua is an open-source scripting language. It is fast and leightweight and was used mainly in embedded systems and clients.\footnote{\href{http://www.lua.org/about.html}{Reference}} \\
Lua is not object-oriented but due to its tables and meta tables can be easily adopted for multiple purposes. It is a dynamically typed language. It supports eight basic data types: nil, boolean value, number, userdata, function, thread, string, and table.\footnote{\href{http://www.lua.org/pil/2.html}{Reference}} Classic data structures such as Arrays can be represented with Luas tables. \\
Lua is used in MediaWiki with the \textit{Scribunto} extension. Scribunto ``[p]rovides a framework for embedding scripting languages into MediaWiki pages''\footnote{\href{https://www.mediawiki.org/wiki/Extension:Scribunto}{Reference}}. At the moment it only supports Lua. \\
Scribunto allows MediaWiki users to write their own Lua modules in the \textit{Module} namespace. These can be invoked on any other Wiki Page using \\
\begin{lstlisting}[frame=single] 
{{#invoke:module name|function name|argument}}
\end{lstlisting}

In the case of a module called \textit{Module:AboutTopic} with the function \textit{showData} and a string parameter with an entity Id, calling the module on any Wiki page would look like the following. \\
\begin{lstlisting}[frame=single] 
{{#invoke:AboutTopic|showData|Q42}}
\end{lstlisting}



\end {document}