\subsection{Wikibase}
Wikibase is an open source project, which was developed for and is mainly used by Wikidata. It is a MediaWiki extension. \\
Wikibase consists of two parts: \textit{Wikibase repository} and \textit{Wikibase client}. \textbf{Wikibase repository} (in the following text only ``repository'')  stores and manages structured, non-relational, and editable data \footnote{\href{http://wikiba.se/}{reference}}. \\
\textbf{Wikibase client} (in the following text only ``client'') enables the retrieving of a repository's data on a wiki (ibidem). The interlanguage links (``navigation links that show up in the sidebar in most MediaWiki skins which connect an article with related articles in other languages within the same Wiki family''\footnote{\href{https://www.mediawiki.org/wiki/Interlanguage_links}{Reference}}) for example are pulled from the repository via a shared changes database table and a maintaince script and then are sorted by the client's settings\footnote{\href{https://www.mediawiki.org/wiki/Extension:Wikibase_Client}{Reference}}. \\
In the case of Wikidata and Wikipedia, Wikidata is a repository and any Wikipedia connected to Wikidata is a client.