\section{Languages online}

As evident in Figure~\ref{fig:w3techLang} and Figure~\ref{fig:listLang}, by far the most common language online is English. 53.8\% of the content online is in this language, followed by Russian, German and Japanese. \citep{w3techLang} \\
\\
Wikipedia is little different to other websites in this regard. English, Swedish, German, Dutch, French and Russian are among the top ten biggest Wikipedias in terms of the numbers of articles as can be seen in Figure~\ref{fig:wikipedias-articles}. \\
\begin{figure}[H]
\begin{center}
	\begin{tabular}{| l | c | r |}
		\hline			
		Number of articles & Number Wikipedias \\ \hline
		over 5.000.000 & 1 \\
		1.0000.000 -- 4.000.000 & 11 \\
		1.000 - 1.000.000 & 227 \\
		0- 1.000 & 48 \\
		\hline  
	\end{tabular}
	\end{center}
	\caption{Number of articles and number Wikipedias}
	\label{fig:tableNumWP}
\end{figure}
%\caption{Number of articles and number of Wikipedias}
The English Wikipedia is the only one with over five million articles. In total, eleven Wikipedias consist of between one and four million articles. Even though there are only 48 Wikipedias featuring between zero and a thousand articles, there are 227 Wikipedias with between one thousand and one million articles, as visualized in Figure~\ref{fig:tableNumWP}. \citep{wiki:30} \\
\\
When compared to the numbers of first-language speakers a huge gap is clearly visible: The most widespread languages are Chinese, Spanish, English, and Hindi. Hindi, for example, is only the 57th biggest Wikipedia however.
\begin{quote}
``Using the benchmark of 100,000 Wikipedia pages in any given language, it [the 2015 Broadband Report] found that only 53 percent of the world's population has access to sufficient content in their native language to make use of the Internet relevant'' found \citet{atlanticLang}.
\end{quote}
One can conclude that there are only a few languages that are very well resourced, while a vast majority of speakers of other languages does not have access to sufficient information in their native language.