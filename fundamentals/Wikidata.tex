\subsection{Wikidata}
Wikidata is the free and open database by the Wikimedia movement. Similar to the Wikipedia it is user editable. Wikidata collects data on different topics like people, places, events and many more. Due to the structure of Wikidata, the data is multilingual. \\
\todo{include graphic about item structure} Every item has a unique identifier, starting with a "Q". Therefore it is --eindeutig identifizierbar--. Statements are one statement on this item, like the number of inhabitants of a city. Those contain a property, which is the core part and indicating, what kind of statement is made. One or more values are attached to this property. These can be other items, string, quantities, names of Wikimedia Commons files and in the future also identifier of other databases.  \\
The data is under a CC-0 licence and therefore free to anyone to copy, use and distribute. 
The aim is not only to have a database that supports the Wikipedia but can also be used in more contexts. 