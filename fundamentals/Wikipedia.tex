\section{Wikipedia}
Wikipedia is one of the largest websites in the world with articles in over 250 languages. As of January 2016, the English Wikipedia alone amounts to 5,000,000 articles. \citep{wiki:32} \\
The concept of Wikipedia is fairly simple: An open encyclopaedia anyone can edit. It was launched on January 15, 2001. \citep{wiki:31} \\
Each of the over 250 languages have their own Wikipedia, which is linked with \textit{interlanguage links} to the other language Wikipedias. \\
Wikipedia can be edited with an account or without, recording only the editor's IP address in the edit history of an article. \\
Editors can edit Wikipedia via the \textit{Visual Editor}, which provides them with a WYSIWYG interface. WYSIWYG is according to \citet{wysiwyg} ``[d]enoting the representation of text on-screen in a form exactly corresponding to its appearance on a printout.'' The traditional way of editing on Wikipedia is writing content in a markup language developed for Wikipedia and its software, MediaWiki. This markup language is called \textit{wikitext}. \\
Wikipedia is organized by volunteers. To organize and maintain the various aspects of Wikipedia, the communities can nominate \textit{administrators}.
As \citet{wiki:10} states, administrators are
\begin{quote}
 Wikipedia editors who have been granted the technical ability to perform certain special actions on the English Wikipedia, including the ability to block and unblock user accounts and IP addresses from editing, edit fully protected pages, protect and unprotect pages from editing, delete and undelete pages, rename pages without restriction, and use certain other tools.
\end{quote}

To decide whether a topic is inside the scope of Wikipedia and therefore should have an article, the editors of each Wikipedia develop their own notability criteria. General notability criteria on the English Wikipedia can be found and as an article on the Wikipedia. \citet{wiki:11} includes the following. 
\begin{quote}
 If a topic has received significant coverage in reliable sources that are independent of the subject, it is presumed to be suitable for a stand-alone article or list.
\end{quote}

Part of every Wikipedia are \textit{SpecialPages}.Those ``are pages that have no wikitext, but are generated by the software on demand. They are found in the "Special:" namespace'', states \citet{wiki:12}.

\todo[inline]{overall [citation needed]}

