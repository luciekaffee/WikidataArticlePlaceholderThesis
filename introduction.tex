\section{Introduction}
One of the biggest barriers for accessing knowledge on the Internet is language. We tend to provide information in one or at most a few languages, which makes it hard for speakers of all the other languages to access that same information. This is also an issue on Wikipedia, a project widely and internationally used by all kind of people. But there are many topics that are only covered in few languages on Wikipedia. People who do not speak any of these do not have access to all the information available potentially vital to them. This is a huge issue we need to address. \\

Today we have Wikidata, Wikimedia’s knowledge base, which collects multilingual open structured data in one central place and makes it available to everyone. 
The goal of the Article Placeholder extension I was working as the subject of this thesis is to give more people more access to more knowledge by making use of Wikipedia’s reach and Wikidata’s multilingual data.

\todo{very short description: What are AP?}

\href{https://www.fosdem.org/2016/schedule/event/increasing_access_to_free_and_open_knowledge_for_speakers_of_underserved_languages_on_wikipedia/}{FOSDEM talk}