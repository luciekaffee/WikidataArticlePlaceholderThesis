\chapter{Introduction}

\todo[inline]{Add dates to tables, actually add tables to Appendix}
One of the biggest barriers for accessing knowledge on the Internet is language. The tendency is to provide information in one or at most a few languages, which makes it hard for speakers of all other languages to access that same information. This is also an issue with Wikipedia, a project used all across the globe by all kinds of people. There are many topics that are only covered in a few languages on Wikipedia. People who do not speak these languages do not have access to all the information available and potentially vital to them. This is an important issue that needs to be addressed. \\
As visible in Figure~\ref{fig:w3techLang} and Figure~\ref{fig:listLang}, the by far most common language online is English. 53.8\% of the content online is in this language, followed by Russian, German and Japanese. \citep{w3techLang} \\
\\
A similar picture can be seen when looking at Wikipedia. English, Swedish, German, Dutch, French and Russian are among the top ten biggest Wikipedias in terms of the numbers of articles as visible in Figure~\ref{fig:wikipedias-articles}. \\
\begin{center}
\begin{tabular}{| l | c | r |}
  \hline			
  Number of articles & Number Wikipedias \\ \hline
  over 5.000.000 & 1 \\
  1.0000.000 -- 4.000.000 & 11 \\
  1.000 - 1.000.000 & 227 \\
  0- 1.000 & 48 \\
  \hline  
\end{tabular}
\end{center}

The English Wikipedia is the only one with over five million articles. There are eleven Wikipedias with between one and four million articles. Even though there are only 48 Wikipedias with between zero and a thousand articles, there are 227 Wikipedias with between one thousand and one million articles. \citep{wiki:30} \\

When compared to the numbers of first-language speakers a huge gap is clearly visible: The most widespread languages are Chinese, Spanish, English, and Hindi. Hindi, for example, is only the 57th biggest Wikipedia however.
\begin{quote}
Using the benchmark of 100,000 Wikipedia pages in any given language, it found that only 53 percent of the world’s population has access to sufficient content in their native language to make use of the Internet relevant.
\end{quote} 
found \citet{atlanticLang}.
The conclusion can be drawn that there are a few languages that are very well resourced, while a fast majority of speakers of other languages do not have access to much information in their native language. \todo{Quote that this one of the mayor problems for access} \\
\\
To close this informational gap, the ArticlePlaceholder extension was developed. The aim is to give more people more access to more knowledge by making use of Wikipedia’s reach and Wikidata’s multilingual data. \todo{Short intro to Wikidata}\\
ArticlePlaceholder generates content pages on Wikipedia in the community's language from data and offer the possibility to create actual articles and translate them. It aims to support readers as well as editors. \\
\\
In the course of this thesis, there will be an introduction to the fundamentals needed for the project, an overview of the previous work done with regards to Wikidata and Wikipedia, the requirements, that should be fulfilled for the project, and the implementation of the technical aspects of the extension. 