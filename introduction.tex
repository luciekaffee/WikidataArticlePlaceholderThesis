\section{Introduction}
One of the biggest barriers for accessing knowledge on the Internet is language. The tendency is to provide information in one or at most a few languages, which makes it hard for speakers of all other languages to access that same information. This is also an issue with Wikipedia, a project \todo{widely and internationally?} widely and internationally used by all kinds of people. There are many topics that are only covered in few languages on Wikipedia. People who do not speak these do not have access to all the information available and potentially vital to them. This is a huge issue that needs to be addressed. \\
\\
Today there are Wikidata, Wikimedia’s knowledge base, which collects multilingual structured data in one central place and makes it publicly available under a free license. \\
The goal of the Article Placeholder extension  is to give more people more access to more knowledge by making use of Wikipedia’s reach and Wikidata’s multilingual data.

\todo{very short description: What are AP?}

\href{https://www.fosdem.org/2016/schedule/event/increasing_access_to_free_and_open_knowledge_for_speakers_of_underserved_languages_on_wikipedia/}{FOSDEM talk}