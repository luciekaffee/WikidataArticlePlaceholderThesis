\section{Introduction}
One of the biggest barriers for accessing knowledge on the Internet is language. The tendency is to provide information in one or at most a few languages, which makes it hard for speakers of all other languages to access that same information. This is also an issue with Wikipedia, a project used all across the globe by all kinds of people. There are many topics that are only covered in a few languages on Wikipedia. People who do not speak these do not have access to all the information available and potentially vital to them. This is a huge issue that needs to be addressed. \\
As visible in the \textit{Usage Statistic for Content Languages for Websites}, the by far most common language online is English with 53.8\% of the content online being in this langauge, followed by Russian, German and Japanese. (See Appendix for a table of Usage Statistic for Content Languages for Websites) \\
A similar image can be detected when looking at Wikipedia. English, Swedish, Cebuano, German, Dutch, French and Russian are the article-wise seven biggest Wikipedias. \\
The English Wikipedia is the only one with over five million articles. There are eleven Wikipedias with one two four million articles. Even though there are only 48 Wikipedias with zero to a thousand articles, there are 227 Wikipedias with one thousand to one million articles. (See Appendix for a table of Number of Wikipedias per Article) \\
When compared to the numbers of first-language speakers of languages a huge cap is clearly visible: The most widespread languages are Chinese, Spanish, English and Hindi. Hindi for example is only the 57th biggest Wikipedia though. \\
The conclusion can be drawn that there are a few languages, that are very well resourced while a fast majority of speakers of other languages do not have access to a lot of information in their first language. \todo{Quote that this one of the majour problems for access} \\
\\
To close this informational gap the ArticlePlaceholder extension is developed. The aim is to give more people more access to more knowledge by making use of Wikipedia’s reach and Wikidata’s multilingual data. \\ ArticlePlaceholder generate content pages on Wikipedia in the community's languge from data and offer the possibility to create actual articles and translate them. This will support readers as well as editors. \\
\\
In the course of this thesis, there will be an introduction to the fundamentals needed for the project, an overview of the previous work done in the field in regards of Wikidata and Wikipedia, the requirements, that should be fulfilled for the project and the implementation of the technical aspects of the extension. 