\chapter{Introduction}

One of the greatest barriers for accessing knowledge on the Internet is language. There is a tendency to provide information in one or at most a few languages, which makes it hard for speakers of all other languages to access that same information. This is also an issue with Wikipedia, a project used all across the globe by all kinds of people. There are many topics that are only covered in a few languages on Wikipedia. People who do not speak these languages do not have access to all the information available and potentially vital to them. This important issue needs to be addressed. \\
\\
The ArticlePlaceholder extension was developed to close this informational gap. The objective is to give more people more access to knowledge by making use of Wikipedia’s reach and Wikidata’s multilingual data. \todo{Short intro to Wikidata}\\
ArticlePlaceholder generates content pages on Wikipedia in the community's language from data and offers the possibility to create actual articles and to translate them. The extension aims at supporting readers as well as editors. \\
\\
In the course of this thesis, an introduction to the fundamentals needed for the project will be given, as well as an overview over the previous work done with regards to Wikidata and Wikipedia. \\
Since a lot of the development was user-centric, personas were created to develop scenarios and user stories. From those scenarios non-functional and functional requirements were derived. The implementation realizes many of the requirements and includes the technical aspects of the extension.
