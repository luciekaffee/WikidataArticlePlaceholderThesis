\documentclass[11pt]{article}
\title {{Structure of the thesis}}
\author {Lucie-Aim\'{e}e Kaffee}
\date{}

\begin {document}

\maketitle

\section{Introduction}

\subsection{To Wikipedia}
\subsection{To Wikidata}

\section{Initial sitatuation}

\section{Objective}

\section{Personas}

\begin{itemize}
\item Someone reading a "big" Wikipedia (German, English)
\item Someone editing a Wikipedia
\item Someone reading a "small" Wikipedia 
\end{itemize}

\section{Non-functional requirements}

\begin{itemize}
\item User should be able to influence (the display of the item \& the ordering of statements)
\item User should be able to discover the Article placeholder
\item Create an Article from the ArticlePlaceholder
\end{itemize}

\section{Functional requirements}

\subsection[wmflabs]{Deploy on wmflabs}
\begin{itemize}
\item Get to the ArticlePlaceholder via the SpecialPage URL
\item A default display of the data
\item Localisation of the whole extension
\item Show the article connected to the item, if an article on the wiki exists
\item Get to the Article Placeholders via search
\item A possibility to create an article from scratch from an Article Placeholder by giving the user the possibility to enter a title
\item Show main image of the item seperate from the statement list
\end{itemize}

\subsection[betafeature]{Deploy as beta feature}
\begin{itemize}
\item Nice default display
\item Only notable items in the search
\item Proper name for the special page
\item Statement groups sorted
\end{itemize}

\subsection[deploy]{Deploy on a Wikipedia}

\section{User stories}

\section{Implementation}

\end {document}