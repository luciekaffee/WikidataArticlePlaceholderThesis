\chapter{Personas}

According to \citet[182]{design:01} personas are created to ``understand and perfectly meet the needs of the critical few [rather] than to poorly meet the needs of many''. To meet the needs of the users that the ArticlePlaceholder extension is intended for, personas were developed based on the existing research of Wikipedia users and the more general behavior of people in different countries. From these personas, scenarios and user stories were developed to derive functional and non-functional requirements that were used to implement the extension. \\
\\
\section{Rashidi: Reader of a small Wikipedia}
Rashidi (29) lives in Kaduna in north-west Nigeria. His native language is Hausa, which is ``a major world language with more first-language speakers than any other sub-Saharan African language---an estimated 30 million or more---most of whom live in northern Nigeria and southern areas of the neighbouring Republic of Niger'' \cite[p1]{hausa}. \\
Hausa Wikipedia however has only 1,356 pages. (See Appendix for a table of Hausa Wikipedia Statistic) Rashidi knows, Hausa Wikipedia exists but never actually uses it. \\
He is a musician and is very interested in music theory. \\
He accesses the Internet mostly on his phone. When he searches for information, he uses Google. But since there is little content in his language that he finds interesting, he rarely uses the Internet to research information but rather talks with friends or looks things up in journals and books.


\section{Edha: Editor of a small Wikipedia}
Edha (27) is an architect in Bhubaneswar, Odisha, India. Her native language is Odia. ``Odia is an Indo-Aryan language spoken by about 33 million people mainly in the Indian state of Odisha, and also in West Bengal, Jharkhand, and Gujarat.'' \citep{odia} She speaks English also.\\
Edha started contributing to the English Wikipedia in 2014 and then decided to write her first article in Odia on a topic related to her field of work. She speaks both Odia and English, fluently, and in consequence translates many articles. \\
Odia Wikipedia has a very small community. There are hardly any people editing, since the Wikipedia has only a handful of articles and few people are aware that a Wikipedia in their language exists. Most people prefer to read articles on bigger Wikipedias, and she is occasionally frustrated by the scant recognition her work gets due to the limited number of readers. \\
She does not know any programming but is very confident with a computer because of her time at the university and the fact that she outlines most of her architectural plans on the computer at work. \\
At home she uses a smartphone extensively to read news articles and watch movies. She edits Wikipedia articles on her smartphone, as she is only able to access a computer at her workplace. \\  
Beside architecture she is mainly interested in art. She enjoys spending her free time reading about art history and going to galleries. \\
She got involved in Wikipedia through a friend who was already an editor and helped her during her first steps.

\section{Julian: Editor of a big Wikipedia}
Julian (17) is an active editor of Wikipedia. He is a German pupil attending his last year of high school before going on to study at university. He is also involved in various software projects as a developer,  and spends his free time programming, reading books and watching TV series. \\
He knows his way around computers pretty well and enjoys working with them. \todo{Too much ``he''} \\
Julian is an active Wikipedia editor and has been involved since 2009. His reason for contributing is his wish to improve the quality of articles and provide better access to free knowledge for everyone. \\
Even though he is a long-time contributor, he does not feel part of the German Wikipedia community. He has never felt comfortable with the way people discuss articles. The discussion culture is the only area of Wikipedia that could be improved in his opinion so as to make it more inclusive and less elitist. \\
Nonetheless, he was part of the project on Wikipedia for junior editors (\textit{Jungwikipedianer}), knows many other editors, and enjoys discussing the most recent Wikipedia news with them in private. \\
His interests are mostly related to computer science and philosophy, and uses Wikipedia to get a deeper understanding of these topics. \\
He mostly edits existing articles but has also created his own articles where topics had yet to be covered.

\section{Catrin: Reader of a big Wikipedia}
Catrin (46) is a social worker living in Paris, France. She works with children and adolescents, and gives lectures at a university on child protection. In her scant free time she enjoys reading and going to museums. With her children majoring in different fields, she enjoys keeping up to date with their topics too. \\
She is a native speaker of French and does not speak any other languages. \\
In early 2000 she taught children and the elderly proper use of a computer and the Internet and therefore knows the basics of those topics. She has not caught up with the latest technologies however. \\
She accesses the Internet at home from her phone. \\
Although in the past she would spend a lot of time in the library, nowadays due to limited time she acquires most of her knowledge online. \\
She has never edited Wikipedia but uses the project to read up on the topics she is interested in. She uses Wikipedia primarily to get an overview of a topic, and uses the references for further research. \\
She enjoys the project, and understands the structure and the idea of anyone being able to edit, even though she is not contributing to the articles herself. \\


\section{Heather: Wikidata editor}
Heather (32) is a US-American software developer. She is mainly involved with the \textit{OpenStreetMap} project and spends much of her time contributing to open source projects or organizing meetups for projects she is contributing to. \\
She mainly contributes to English Wikipedia, and has written many articles over the past eleven years. \\
She has known of Wikidata from the start of the project and was one of the first contributors. She has a collection of bots running on Wikidata, which add data on topics she is interested in. \\
The thing that bothers her the most on Wikidata are missing references, which are even now often not added automatically but manually by editors. For edits on the page without bots she uses the Wikidata games frequently, which are a ``set of ''games`` that help to improve Wikidata'' \citep{wikidatagame}.
