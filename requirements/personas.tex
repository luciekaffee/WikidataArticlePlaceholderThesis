\chapter{Personas}

According to \citet[182]{design:01} personas are created to ``understand and perfectly meet the needs of the critical few [rather] than to poorly meet the needs of many''. To meet the needs of users the ArticlePlaceholder extension is aiming at, personas were developed based on the existing research of Wikipedia users and more general behavior of people in different countries. From these personas, scenarios and user stories were developed to derive non-functional and functional requirements that were used to implement the extension. \\
\\
\section{Rashidi: Reader of a small Wikipedia}
Rashidi (29) lives in Kaduna in north-west Nigeria. His native language is Hausa, which is ``a major world language with more first-language speakers than any other sub-Saharan African language---an estimated 30 million or more---most of whom live in northern Nigeria and southern areas of the neighbouring Republic of Niger'' finds \citet[1]{hausa}. \\
Hausa Wikipedia however has only 1,356 pages. (See Appendix for a table of Hausa Wikipedia Statistic) Rashidi knows, Hausa Wikipedia exists but never actually uses it. \\
He is a musician and is very interested in the topic. \\
He accesses the Internet mostly on his phone. When he searches for information, he uses Google. But since there is little content in his language that is interesting for him, he rarely uses the Internet to research information but rather talks with friends or looks things up in journals and books.


\section{Edha: Editor of a small Wikipedia}
Edha (27) is an architect in Bhubaneswar, Odisha, India. Her native language is Odia. ``Odia is an Indo-Aryan language spoken by about 33 million people mainly in the Indian state of Odisha, and also in West Bengal, Jharkhand, and Gujarat.'' \citep{odia} She speaks English as well.\\
Edha started contributing to the English Wikipedia in 2014 and then decided to write her first article in Odia on a topic related to her field of work. She speaks both Odia and English, fluently. Therefore she translates a lot of articles. \\
Y Wikipedia has a very small community. There are hardly any people editing since the Wikipedia has only a handful of articles and few people know a Wikipedia project in their language exists. Most people rather read articles on bigger Wikipedia projects and she is occasionally frustrated by the little recognition her work gets due to the limited number of readers. \\
She does not know any programming but is very confident with a computer because of her time at the university and the fact that she sketches most of her plans \todo{improve this!} on the computer at work. \\
At home she uses a smart phone a lot to read news articles and watch movies. She edits Wikipedia articles on the smart phone, as she is only able to access a computer at her work place. \\  
Beside architecture she is mainly interested in art. She enjoys spending her free time reading about art history and going to galleries. \\
She got involved in Wikipedia through a friend who already was an editor and helped her during her first steps.

\section{Julian: Editor of a big Wikipedia}
Julian (17) is an active editor of Wikipedia. He is a German pupil and attends his last year of high school before he goes on to study at university. He is also involved in various software projects as a developer. In his free time he spends time programming, reading books and watching TV series. \todo{A lot of He in this section. try to rephrase in order to avoid monotony.}\\
He knows his way around computers pretty well and enjoys working with them. \\
He is an active Wikipedia editor and has been around since 2009. His reason for contributing is his wish to improve the quality of articles and provide better access to free knowledge for everyone. \\
Even though he is a long-time contributor, he does not feel as part of the German Wikipedia community. He never felt comfortable with the way people discuss. The discussion culture is the only area of Wikipedia that could be improved in his opinion in order to be more inclusive and less elitist. \\
He was part of the project on Wikipedia for under-aged editors ("Jungwikipedianer") anyway and knows a lot of other editors and enjoys discussing with them about the most recent Wikipedia news in private. \\
His interests are mostly related to computer science and philosophy, where he uses Wikipedia to get a deeper understanding of these topics. \\
He mostly edits existing articles but has created his own articles in the past, where topics had yet to be covered.

\section{Catrin: Reader of a big Wikipedia}
Catrin (46) is a social worker living in Paris, France. She is working with children and adolescents and gives lectures at a university on working with children. In the little free time she has she enjoys reading and going to museums. Due to her children majoring in different fields she enjoys keeping up to date with their topics, too. \\
She is a native speaker of French and does not speak any other languages. \\
In early 2000 she taught children and the elderly proper handling of a computer and the internet and therefore knows the basics of those topics. She didn't catch up with the latest technologies, though. \\
She uses a computer mostly working at the university and accesses the Internet at home from her phone. \\
Even though she used to spend a lot of time in the library due to limited time she acquires knowledge mostly online nowadays. \\
She has never edited Wikipedia but uses the project to read up on the topics she is interested in. She mainly uses Wikipedia to get an overview of a topic and uses the references for further research. \\
She enjoys the project and does understand the structure and the idea of anyone being able to edit even though she does not do it herself. \\


\section{Heather: Wikidata editor}
Heather (32) is a US-American software developer. She is mainly involved with the Open Street Map project and spends a lot of her time contributing to open source projects or organizing meetups for projects she is contributing to. \\
She mainly contributes to the English Wikipedia and has written lots of articles over the past eleven years. \\
She knew about Wikidata from the start of the project and was one of the first contributors. She has a collection of bots running on Wikidata, which add data on topics she is interested in. \\
The thing that bothers her the most on Wikidata are missing references, which are to date often not added automatically but manually by editors. For edits on the page without bots she uses the Wikidata game a lot. \todo{explain/ link WD game}
