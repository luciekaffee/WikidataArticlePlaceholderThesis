\documentclass[11pt]{article}

\usepackage{todonotes}

\begin {document}

\subsection{Personas}

\subsection{Reader of a small Wikipedia}

\subsection{X: Editor of a small Wikipedia}
X (27) is an architect in Y. She started contributing to English Wikipedia in 2014 and then decided to write her first article in Y on a topic related to her field of work. She speaks both, Y and English, fluently. Therefore she translates a lot of articles. \\
Y Wikipedia has a very small community. There are very few people editing since the Wikipedia has very little articles and few people know a Wikipedia project in their language exists. Most people read articles rather on bigger Wikipedia projects and she is sometimes frustrated by the little recognition her works gets due to the limited number of readers. \\
She does not know any programming but is very confident with her computer due to her university time and her work, where she sketches most of her plans on the computer. \\
At home she uses her smart phone a lot to read news and watch movies. She does edit on the smart phone, too due to only be able to access a computer at her work place. \\  
Beside architecture she is mainly interested in art. She enjoys spending her free time reading about art history and going to galleries. \\
She got involved in Wikipedia through a friend who was already an editor and helped her with her first steps.

\subsubsection{Julian: Editor of a big Wikipedia}
Julian (17) is an active editor of Wikipedia. He is a German pupil and is in his last year of high school before studying at a university. He is also involved in various software projects as a developer. In his free time he spends time programming, reading books and watching series. \\
He knows his way around computer pretty well and enjoys working with them. \\
He is an active Wikipedia editor since 2009. He started editing for the same reason he is now still involved: He wants to improve the quality of articles and provide better access to free knowledge for everyone. \\
Even though he is a long-time contributor, he does not feel he is a part of the German Wikipedia community. He never felt comfortable with the way people discuss. The discussion culture is the the only part where he thinks Wikipedia could improve more to be more inclusive and less elitist. \\
He was part of the project on the Wikipedia for under-aged editors ("Jungwikipedianer") anyway and knows a lot of other editors and enjoys discussing with them about the most recent Wikipedia news in private. \\
His interests are mostly related to computer science and philosophy, where he uses Wikipedia to get an overview on these topics. \\
He mostly edits existing articles but did create his own articles in the past, too, if he was interested in a topic that was not covered yet.

\subsubsection{Catrin: Reader of a big Wikipedia}
Catrin (46) is a social worker living in Paris, France. She is working with children and adolescents and gives lectures at a university on working with children. In the little free time she has she enjoys reading and going to museums. Due to her children majoring in different fields she enjoys keeping up to date with their topics, too. \\
She is a native French speaker and speaks fluently English \todo{better wording}. She does know some basics in Latin and Spanish, too, but rarely uses those languages. \\
In early 2000 she taught children and the elderly proper handling of computer and the internet and therefore knows the basics of technology. She didn't catch up with the latest technologies though. \\
She uses her computer mostly at work in the university and access the Internet at home from her phone. \\
Even though she used to spend a lot of time in the library due to limited time she acquires knowledge mostly online nowadays. \\
She never edited Wikipedia but does use the project a lot to read up on the topics she is interested in. She mainly uses Wikipedia to get an overview on a topic and uses the references for a better insight in a certain topic. \\
She enjoys the project and does understand the structure and the idea of anyone being able to edit even though she doesn't do it. \\


\subsection{Wikidata editor/Heather}
Heather (32) is an US-American software developer. She is mainly involved with the Open Street Map project and spends a lot of her time contributing to open source projects or organizing meetups for projects she is contributing to. \\
She is contributing mainly to English Wikipedia since eleven years and has written a lot of articles. 
She knew about Wikidata from the time the project started and was one of the first contributors. She has a differing amount of bots running on Wikidata, which add data on topics she is interested in. \\
The thing that bothers her the most on Wikidata are missing references, which are difficult to add automated and need the manual work of editors. For manual edits she uses the Wikidata game a lot. \todo{explain/ link WD game}

\begin{itemize}
\item Job
\item goals
\item desires
\item Education/Knowledge
\item knowledge about Computer
\item attitude towards the project/Wikipedia
\item Hobbies
\item expectations
\item restrictions

\item
\item Someone reading a "big" Wikipedia (German, English)
\item Someone editing a "big" Wikipedia
\item Someone editing a "small" Wikipedia
\item Someone reading a "small" Wikipedia 
\item Wikidata editor
\item Troll
\end{itemize}

\end {document}
