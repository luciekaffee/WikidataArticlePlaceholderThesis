\documentclass[11pt]{article}

\usepackage[dvipsnames]{xcolor}
\usepackage{hyperref}
\usepackage{todonotes}
\usepackage{listings}

\title {{Functional requirements}}
\author {Lucie-Aim\'{e}e Kaffee}
\date{}

\begin {document}

\listoftodos

\section {Functional requirements}
The functional requirements are based on the non-functional requirements and aim to implement those.
There are also requirements, that are based on the circumstances and the techniques used. \\
To be able to use the data of Wikidata, ArticlePlaceholder needs to base on Wikibase and mediawiki.
As mediawiki is implemented in PHP, the main part of the extension needs to make use of the language as well as the services provided by mediawiki and Wikibase. \\
The language for scripts on the Wikipedia is Lua. Therefore the parts that will be user-editable need to be written in Lua so the display can be overwritten by the local communities. \\
The default for the content pages should present the information and data in a useful way and make use of the images in Wikimedia Commons, that are connected to the item displayed.\\
The content pages generated by the extension will need to work on mobile. \\
\\
As described before, the requirements would differ from the different stages in the deployment cycle. They are always built on top of each other but since the parts have different importance we would focus on a very basic set up first and add up more features from there. Here are the requirements described as they would be in the final system deployed to a Wikipedia.

\todo[inline, color=green!40]{ Text: In general; Link to other placeholder }%

\subsection{Naming}
The special page should have a short but easy to remember. Calling the special page "Article Placeholder" would have been confusing due to people expecting it to generate a whole article then. Therefore it is called "AboutTopic".
\subsection{Display the article placeholder}
For a first setup, it is necessary to give a first insight into what the extension is actually doing. Therefore, the main task is to have a basic default display of the statements. They don't need to look very pretty or easy to understand yet, they only need to make clear, that it is possible to display data from Wikidata and shouldn't look too much like an content page a user would create manually. As part of this display, it is already important to separate the main image of the item separated from the statement list as a first hint, that there is actually being more work being done on the design part. \\
\todo[inline, color=green!40]{ Text: Display; identifier, every part user-editable }%
\subsection{Localisation}
Of course it is important to have the whole extension localised from the beginning. Not only would it be hard to adjust it in the end but mainly this is the most important part about it- the extension being used by many international communities. Since the data itself is translated in Wikidata, this focuses more on things like labels on buttons, the name of the special page itself and related. \\
\subsection{Discover article placeholder}
The first step here is to be able to get to the article placeholder not only over the interface of the special page but also via a handy URL, that gets the item number as a parameter. Basing on the defaults of mediawiki and Wikibase, it would look like this: \colorbox{Gray}{\lstinline[basicstyle=\ttfamily\color{white}]|Special:AboutTopic/Q5279|} \\
\\
A user should be able to find article placeholder while looking for a certain topic. Therefore, it should be integrated in the search on the Wikipedia they are looking at. It can only be attached on the bottom of the results or placed to the right of the regular search results. \todo[color=green!40]{ Limitation of results }%
\todo[inline, color=green!40]{ Text: Discover; red link, search engines }%

\subsection{Creation of an article from the article placehoder}
\todo[inline, color=green!40]{ Text: Create Article; JS button, enter title, noJS }%

\subsection{Sorting of statement groups}

\todo[inline, color=green!40]{ Text: Statement groups; Ordering, limiting of statement groups }%

\subsection{ Connecting the article placeholder }
\todo[inline, color=green!40]{ Text: Connecting; Link existing articles, Link from AP to AP (limitation) }%

\subsection {Identifier}

\subsection {Unit tests}



\begin{itemize}
\item A default display of the data -- 
\item Show main image of the item separated from the statement list -- 
\item Localisation of the whole extension -- 
\item Show the article connected to the item, if an article on the wiki exists
\item Get to the ArticlePlaceholder via the SpecialPage URL --
\item Get to the Article Placeholders via search
\item A possibility to create an article from scratch from an Article Placeholder by giving the user the possibility to enter a title

\item Create an Article from the ArticlePlaceholder
\end{itemize} 
 

\end {document}