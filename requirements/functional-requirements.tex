\documentclass[11pt]{article}

\usepackage[dvipsnames]{xcolor}
\usepackage{hyperref}
\usepackage{todonotes}
\usepackage{listings}
\usepackage{soul}

\title {{Functional requirements}}
\author {Lucie-Aim\'{e}e Kaffee}
\date{}

\begin {document}

\listoftodos

\section {Functional requirements}
The functional requirements are based on the non-functional requirements and aim to implement those.
There are also requirements, that are based on the circumstances and the techniques used. \\
To be able to use the data of Wikidata, ArticlePlaceholder needs to base on Wikibase and mediawiki.
As mediawiki is implemented in PHP, the main part of the extension needs to make use of the language as well as the services provided by mediawiki and Wikibase. \\
The language for scripts on the Wikipedia is Lua. Therefore the parts that will be user-editable need to be written in Lua so they can be overwritten by the local communities. \\
The default for the content pages should present the information and data in a useful way and make use of the images in Wikimedia Commons, that are connected to the item displayed.\\
The content pages generated by the extension will need to work on mobile. \\
\\
As described before, the requirements would differ from the different stages in the deployment cycle. They are always built on top of each other but since the parts have different importance we would focus on a very basic set up first and add up more features from there. Here are the requirements described as they would be in the final system deployed to a Wikipedia.

\todo[inline, color=green!40]{ Text: In general; Link to other placeholder }%

\subsection{Naming}
The special page should have a short name that is easy to remember. Calling the special page "Article Placeholder" would be confusing due to people expecting it to generate a whole article in that case. Therefore it is called "AboutTopic".

\subsection{Display the article placeholder}
For the display of the items, it is necessary to focus on a clear differentiation between the layout of the placeholder and what a user-written article would look like but still keep elements of the well-known Wikipedia layout. \\
Therefore, the main image of the item is on the right side of the page where they are in articles, too.\\
The main part of the page are the statement groups. Those consist of a property, one or multiple values, their qualifier and references. \\
The statement groups are each in a box with a black border and arranged in a tile layout. Depending on the width of the screen, the amount of statement boxes per row differs. \\
The label of the property is the header for each group separated from the rest by a horizontal line. Multiple values for the same statement are also separated by a line. Their qualifiers are right under the corresponding value and in a smaller font. \\
To correspond with the existing layout of Wikipedia, the references are in an own section on the bottom of the page. \\
The identifiers of an item should be in a distinct place to emphasize them and at the same time distant them from the other statements. They are listed in underneath the main image of the article to the right of the statement groups. \\
The described layout should be the default but it is very important to give the user control over everything and therefore let every part be user editable. The renderer are in Lua and every part has getter and setter to overwrite them and adjust the design to the need of the communities. \\

\subsection{Localisation}
Of course it is important to have the whole extension localised from the beginning. Not only would it be hard to adjust it in the end but mainly this is the most important part about it- the extension being used by many international communities. Since the data itself is translated in Wikidata, this focuses more on things like labels on buttons and the name of the special page itself. \\
\subsection{Discover article placeholder}
\st{The first step here}g is to be able to get to the article placeholder not only over the interface of the special page but also via a handy URL, that gets the item number as a parameter. Basing on the defaults of mediawiki and Wikibase, it would look like this: \colorbox{Gray}{\lstinline[basicstyle=\ttfamily\color{white}]|Special:AboutTopic/Q5279|} \\
\\
A user should be able to find article placeholder while looking for a certain topic. Therefore, it should be integrated in the search on the Wikipedia they are looking at. It can only be attached on the bottom of the results or placed to the right of the regular search results. \todo[color=green!40]{ Limitation of results }%
\todo[inline, color=green!40]{ Text: Discover; red link, search engines }%

\subsection{Creation of an article from the article placehoder}
\todo[inline, color=green!40]{ Text: Create Article; JS button, enter title, noJS }%

\subsection{Sorting of statement groups}

\todo[inline, color=green!40]{ Text: Statement groups; Ordering, limiting of statement groups }%

\subsection{ Connecting the article placeholder }
\todo[inline, color=green!40]{ Text: Connecting; Link existing articles, Link from AP to AP (limitation) }%

\subsection {Identifier}

\subsection {Unit tests}



\begin{itemize}
\item A default display of the data -- 
\item Show main image of the item separated from the statement list -- 
\item Localisation of the whole extension -- 
\item Show the article connected to the item, if an article on the wiki exists
\item Get to the ArticlePlaceholder via the SpecialPage URL --
\item Get to the Article Placeholders via search
\item A possibility to create an article from scratch from an Article Placeholder by giving the user the possibility to enter a title

\item Create an Article from the ArticlePlaceholder
\end{itemize} 
 

\end {document}