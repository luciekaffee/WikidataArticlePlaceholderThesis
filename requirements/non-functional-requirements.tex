\documentclass[11pt]{article}

\usepackage{todonotes}

\title {{Non-functional requirements}}
\author {Lucie-Aim\'{e}e Kaffee}
\date{}

\begin {document}

\section{Non-functional requirements}

The extension aims to provide access to previously unavailable information to reader of Wikipedia to let them gather more knowledge. \todo{Improve sentence // better reasoning} \\
The requirements should all follow this main goal. \\
User should be able to discover the Article Placholder generated pages while browsing the Wikipedia or researching a certain topic. \\
While the content pages are auto-generated, editors of the different wiki projects should still feel a sense of ownership over these. They need to keep control over the pages as much as possible to adjust them to their community's needs.  \\
At the same time the extension should provide a default, that makes it attractive and engaging to use. Especially for speakers of the target languages- languages, that are under represented in the Wikipedia and have small communities and a small amount of articles. They often don't even have enough editors to write content so they can't be expected to take the more-effort of being involved with the technical aspects, too.  \\
The pages should engage and encourage editing when appropriate. Therefore they should invite readers to create articles possibly with the information provided by Wikidata. They will need to have an option to translate existing articles from another language the reader may speak, too. \\
Wikipedia is a big and widely used project and has over the years built a reputation as a trust worthy source of knowledge. The extension should maintain this status. \\
The article placeholders must clearly differ from human-written articles in order to make the reader consider the fact that it is auto-generated from pure data when they *bewerten* the content. It is also important due to making clear, that this is not a static, finished page and should motivate the reader to participate where it is appropriate. \\

\end {document}