\chapter{Non-functional requirements}

The extension aims to provide access to previously unavailable information to readers of Wikipedia to allow them access to more knowledge. \\
The requirements should all follow this main goal. \\
The user should be able to discover the Article Placholder generated pages while browsing the Wikipedia or researching a certain topic. \\
While the content pages are auto-generated, editors of the Wikipedia projects should still have a sense of ownership over the pages. They need to keep control over the pages as much as possible to be able to adjust them to their community's needs. \\
At the same time the extension should provide a default that makes it attractive and engaging to use. Especially for speakers of the target languages, that are under represented in Wikimedia projects, have small communities, and a small amount of articles. The small Wikipedias often do not have enough editors to write content so they can't be expected to make the effort of being involved with the technical aspects, too.  \\
The pages should engage and encourage editing when appropriate. Therefore, they should invite readers to create articles - possibly with the information provided by Wikidata. They will need to have an option to translate existing articles from another language the reader may speak, too. \\
Wikipedia is a big and widely used project and has built a reputation as a trustworthy source of knowledge. The extension should maintain this status. \\
The article placeholders must clearly differ from human-written articles in order to make the reader consider the fact that it is auto-generated from pure data when they evaluate the content. It is also important to clarify that this is not a static, finished page and should motivate the reader to participate. \\