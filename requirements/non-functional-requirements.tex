\chapter{Non-functional requirements}

The extension aims to provide access to previously unavailable information to readers of Wikipedia. \\
The requirements should all follow this main objective. \\
The user should be able to discover the ArticlePlaceholder-generated pages while browsing the Wikipedia or researching a certain topic. \\
While the content pages are auto-generated, editors of the Wikipedia projects should still have a sense of ownership regarding the content. They need to keep control over the pages as much as possible, so as to be able to adjust them to their community's needs. \\
At the same time the extension should provide a default that makes it attractive and engaging to use. This is especially true for speakers of the target languages that are under-represented in Wikimedia projects, which have small communities and a small number of articles. The smaller Wikipedias often do not have enough editors to write content, so the editors can't be expected to make the effort of being involved with the technical aspects, too.  \\
The pages should engage and encourage editing when appropriate. Therefore, they should invite readers to create articles - possibly with the information provided by Wikidata. In addition, they need to have an option to translate existing articles from any other language the reader may speak. \\
Wikipedia is a large and widely used project, and has built a reputation as a trustworthy source of knowledge. The extension must maintain this status. \\
\\
The ArticlePlaceholders must clearly differ from human-written articles in order to make the reader aware when evaluating the content that it is auto-generated from pure data. It is also important to clarify that this is not a static, finished page and should motivate the reader to participate. \\