\chapter{User stories}

From the previous requirements following user stories could be identified. The user stories follow the style of the Connextra development team, which is \todo{quote: http://www.strongandagile.co.uk/index.php/what-makes-a-good-user-story/}"As a [user role], I want [function] So that [benefit]"\footnote{\href{http://agilecoach.typepad.com/photos/connextra_user_story_2001/connextrastorycard.html}{ConnextraStoryCard}}. \\

\todo[inline]{Not the wish of the users but our solution- think more from the user perspective!}

As a reader of a small Wikipedia, I want to access readable information from Wikidata so that I can get all the information available in my language. \\
\\
As a reader of a small Wikipedia, I want to search for a topic and get to an article placeholder if there is no article on the topic yet so that I can get information on the topic. \\ As a reader of Wikipedia, I want to search for a topic and find an Article Placeholder if no article exists yet, so that I can access all information available on a topic. \\
\\
As a reader, I want to link to an article paceholder and get to the article if it exists so that I can read the article rather than just see the data. \\
\\
As a reader, I want to see the Article Placeholer of a certain item, so that I can link to the placeholder rather than the Wikidata item. \\
\\
As a reader, I want to see the most important information about a topic first, so that I don't have to read through all the data to get what really interests me. \\
\\
As an editor, I want to influence the order items are shown so that I can adapt it to what is most important for us, which may differ from other communities.\\
\\
As an editor, I want to have an easy way to get from an Article Placeholder to the edit page, so that I can create an article when appropriate. \\
\\
As an editor, I want to have the option to translate articles from an Article Placeholder, so that I am able to contribute to the diversity of articles on my Wiki. \\
\\
As an editor familiar with Lua, I want to be able to adjust the layout of the Article Placeholder on my Wikipedia so that I can adjust it to the needs of my community. \\
\\
As an editor and reader of a very small Wikipedia, I want to have a pleasant default for the Article Placeholder layout and order, so that we are not in need of someone familiar with Lua to make the extension work. \\
\\
As a reader, I want to have a clear distinction between a generated content page and a manual written article, so that I can classify the information in the appropriate context. \\
\\
As a reader of a small Wikipedia, I want the Article Placeholder to be responsive so that I can read on a mobile phone. 