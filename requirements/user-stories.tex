\documentclass[11pt]{article}

\usepackage{tabto}

\usepackage{hyperref}
\usepackage{tikz}
\usetikzlibrary{snakes}

\title {{Functional requirements}}
\author {Lucie-Aim\'{e}e Kaffee}
\date{}

\begin {document}

\subsection{User stories}

\begin{itemize}
\item Someone reading a "big" Wikipedia (German, English)
\item Someone editing a "big" Wikipedia
\item Someone editing a "small" Wikipedia
\item Someone reading a "small" Wikipedia 
\item Wikidata editor
\item Troll

\item Discovery (wie komme ich dahin)\\
-- suche\\
-- über die url der special page\\
-- red links (discussion)\\
-- wenn es Seite schon gibt, zeige diese an\\
\item Display -- default (nicht veränderbar)
-- alle statements vs “nützliche” statements
-- kein importet from as reference
-- ordnen der statements (property suggester?, discussion)
-- klar machen, dass placeholder
\item extended (veränderbar)
-- veränderbare Anzeigemöglichkeit
-- veränderung der Daten (am Anfang mit Link zu Wikidata)
\item Translation of existing content (translation tool) (über sitelinks)
\item encouraging article creation 
\item Creation of article from Article Placeholder
\end{itemize}

\end {document}