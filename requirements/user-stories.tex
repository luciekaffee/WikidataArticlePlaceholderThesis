\chapter{User stories}

From the previous requirements the following user stories could be identified. The user stories follow the style of the Connextra development team (\textit{Connextra format}), which is %\todo{quote: http://www.strongandagile.co.uk/index.php/what-makes-a-good-user-story/}
\begin{quote}
\textit{As a [user role], \newline I want [function] \newline So that[benefit]}
\end{quote}
%\footnote{\href{http://agilecoach.typepad.com/photos/connextra_user_story_2001/connextrastorycard.html}{ConnextraStoryCard}}. \\

The following user stories were derived using the described methods. \\
\\
As a reader of a small Wikipedia, I want to access all information available in my language so that I can use my Wikipedia without disadvantages. \\
\\
As a reader of a Wikipedia, I want to search for a topic and get all information available even if there is no article created yet so that I can read up on the topic. \\
\\
\st{As a reader, I want to link to an ArticlePaceholder and get to the article if it exists so that I can read the article rather than just see the data.} \\
\\
As a reader familiar with Wikidata, I want to see the ArticlePlaceholer of a certain item, so that I can link to the placeholder rather than the Wikidata item. \\
\\
As a reader, I want to see the most important information on a page first, so that I do not have to read through everything to get what really interests me. \\
\\
As an editor, I want to influence the order information is shown in so that I can adapt it to what is most important for my community, which may differ from other communities.\\
\\
As an editor, I want to have an easy way to get from an ArticlePlaceholder to the edit page, so that I can create an article when appropriate. \\
\\
As an editor, I want to have the option to translate articles from an Article Placeholder, so that I am able to contribute to the diversity of articles on my Wikipedia. \\
\\
As an editor familiar with Lua, I want to be able to adjust the layout of the ArticlePlaceholder on my Wikipedia so that I can adjust it to the needs of my community. \\
\\
As an editor and reader of a very small Wikipedia, I want to have a pleasant layout of any content page, so that I do not have to adjust to new styles. \\
\\
As a reader, I want to be able to clearly differentiate between a generated content page and a manually written article, so that I can classify the information in the appropriate context. \\
\\
As a reader of a small Wikipedia, I want any content page to work on every device independent of screen size, so that I can access Wikipedia everywhere. 