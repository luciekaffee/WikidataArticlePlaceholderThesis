\documentclass[11pt]{article}

\usepackage[utf8]{inputenc}
\usepackage[dvipsnames]{xcolor}
\usepackage{hyperref}
\usepackage{todonotes}
\usepackage{listings}
\usepackage{soul}
\usepackage{tikz}
\usetikzlibrary{shapes,arrows,snakes}
\usepackage{courier}
\usepackage{lineno}

\title{Generating Article Placeholders from Wikidata for Wikipedia:\\Increasing access to free and open knowledge}
\author{Lucie-Aim\'{e}e Kaffee}
\date{}

\begin {document}
\section{User stories}

From the previous requirements following user stories could be identified. The user stories follow the style of the Connextra development team, which is \todo{quote: http://www.strongandagile.co.uk/index.php/what-makes-a-good-user-story/}"As a [user role], I want [function] So that [benefit]"\footnote{\href{http://agilecoach.typepad.com/photos/connextra_user_story_2001/connextrastorycard.html}{ConnextraStoryCard}}. \\

As a reader of a small Wikipedia, I want to access readable information from Wikidata so that I can get all the information available in my language. \\
\\
As a reader of a small Wikipedia, I want to search for a topic and get to an article placeholder if there is no article on the topic yet so that I can get information on the topic. \\
\\
As a reader, I want to link to an article paceholder and get to the article if it exists so that I can read the article rather than just see the data. \\
\\
As a reader, I want to see the Article Placeholer of a certain item, so that I can link to the placeholder rather than the Wikidata item. \\
\\
As an editor, I want to have an easy way to get from an Article Placeholder to the edit page, so that I can create an article when appropriate. \\
\\


\begin{itemize}

\item Discovery (wie komme ich dahin)\\
-- suche\\
-- über die url der special page\\
-- red links (discussion)\\
-- wenn es Seite schon gibt, zeige diese an\\
\item Display -- default (nicht veränderbar)
-- alle statements vs “nützliche” statements
-- kein importet from as reference
-- ordnen der statements (property suggester?, discussion)
-- klar machen, dass placeholder
\item extended (veränderbar)
-- veränderbare Anzeigemöglichkeit
-- veränderung der Daten (am Anfang mit Link zu Wikidata)
\item Translation of existing content (translation tool) (über sitelinks)
\item encouraging article creation 
\item Creation of article from Article Placeholder
\end{itemize}

\end{document}