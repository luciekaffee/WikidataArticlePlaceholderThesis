\chapter{Scenarios}

\section{Scenario Rashidi}
Rashidi listens to Jazz on the radio. He really likes Louis Armstrong and wants to look up more information on him. He uses Google to look up information on him and gets to the ArticlePlaceholder on Hausa Wikipedia. He can read up there his date of birth (\textit{4 August 1901}), his place of birth (\textit{New Orleans}) and get more information on, for example, New Orleans and learn, that New Orleans is a city in the United States of America. 
\begin{figure}[H]
	\centering
	\includegraphics[width=\textwidth]{diagrams/UserDiagramRashidiArmstrong.png}
	\caption{Scenario Rashidi}
	\label{fig:ScenarioRashidi}
\end{figure}

\section{Scenario Edha} 
Edha talks with friends about an exhibition they have been to. She hears about Rembrandt, a Dutch painter. As soon as she comes home, she realizes, she forgot to ask which century he lived in and where in the Netherlands he was born. She wants to look the information up on Wikipedia in her language. She opens Wikipedia and enters ``Rembrandt'' in the searchbar. Since there is no article on Rembrandt yet, she finds the ArticlePlaceholder for \textit{``Rembrandt'' (Q5598)} \citep{wd:01} in the result page. Since it also displays the description of the item (\textit{Dutch 17th century painter and etcher})  on the search page next to the name of the author, she can be certain she found the right information. She gets an overview over Rembrandt on the page and can find the place of birth on the ArticlePlaceholder, which is \textit{Leiden}. She doesn't know the place and searches for it on the Wikipedia again. She finds the description \textit{city and municipality in South Holland, Netherlands} \citep{wd:02} on the search page already. That is the information she was looking for, so she goes back to the ArticlePlaceholder. Since in the meantime someone edited the item on Wikidata, she finds even more information on him on the ArticlePlaceholder now. It also tells her his date of birth is \textit{15 July 1606}. \\
In the next days, she researches more on Rembrandt in her local library. With this additional information she decides to create an article. She goes back to the ArticlePlaceholder to get some of the informations and references, that are already there. Then she clicks the button \textit{create an article}, which is labeled in her language. She is able to add the title for the page, which is translated to English \textit{Rembrandt (painter)} to make clear which Rembrandt she means. This title differs from the title of the ArticlePlaceholder, which is only \textit{Rembrandt}. She can enter the information she researched and create an actual article. Since she knows English, she first uses the option to translate the article from the English Wikipedia and adds references in Odia instead of writing a complete new article.
\begin{figure}[H]
	\centering
	\includegraphics[width=\textwidth]{diagrams/UserDiagramEdhaTranslate.png}
	\caption{Scenario Edha}
	\label{fig:ScenarioEdha}
\end{figure}

\section{Scenario Julian}
Julian heard about that ArticlePlaceholder where enabled on German Wikipedia. But since this Wikipedia has quite a lot of articles, he is aware, that he seldom will need them. \\
As part of a presentation in school, he need to research X \todo{Find something actually not existing on German Wikipedia}. He searches for it on Wikipedia and gets to the ArticlePlaceholder. He uses the references provided to the data their. \\
He dislikes the way the data is displayed though. Therefore he wants to adjust the layout. He has contributed to a Module before, so he knows how to write Lua code. He wants to overwrite one of the functions. After having a disscussion with the local community which supports Julian's changes, he adds the code after testing it and the ArticlePlaceholder is adjusted to the community's needs. 
\begin{figure}[H]
	\centering
	\includegraphics[width=\textwidth]{diagrams/UserDiagramJulianLua.png}
	\caption{Scenario Julian}
	\label{fig:ScenarioJulian}
\end{figure}

\section{Scenario Catrin}
One of Catrin's students tells her she used to live in \textit{Kasımpaşa}. She has never heard of it and back at home she wants to research on it on the French Wikipedia. She enters the term in the search bar. Since there is no article on the French Wikipedia, she gets to an ArticlePlaceholder. It doesn't have a description, so she can only see the name. She learns it is an \textit{instance of neighborhood} and is \textit{located in the administrative territorial entity Beyoğlu}. Since she hasn't heard of Beyoğlu yet either, she continues her research and gets to the French Wikipedia article Beyoğlu. There she finds out Beyoğlu is a district of Istanbul, Turkey.
\begin{figure}[H]
	\centering
	\includegraphics[width=\textwidth]{diagrams/UserDiagramCatrinKasimpase.png}
	\caption{Scenario Catrin}
	\label{fig:ScenarioCartin}
\end{figure}

\section{Scenario Heather}

Heather worked on the Wikidata item \textit{``Ada Lovelace'' (Q7259)} over the last days. She added data and mainly references by hand. She wants to see what the item looks like at a Wikipedia, that does not have an article on \textit{Ada Lovelace} yet. \\
She is involved with various small language Wikipedias and even though she doesn't speak the language, she can find the SpecialPage, which will bring her to an ArticlePlaceholder and enters the item ID there to see what it looks like. \\
She is also able to use the item ID to link to the English Wikipedia article as she is redirected to the article. 
\begin{figure}[H]
	\centering
	\includegraphics[width=\textwidth]{diagrams/UserDiagramHeatherSmallWP.png}
	\caption{Scenario Heater, small Wikipedia}
	\label{fig:ScenarioHeatherSmall}
\end{figure}
\begin{figure}[H]
	\centering
	\includegraphics[width=\textwidth]{diagrams/UserDiagramHeatherEnWiki.png}
	\caption{Scenario Heater, English Wikipedia}
	\label{fig:ScenarioHeatherEnWiki}
\end{figure}