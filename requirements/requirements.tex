\documentclass[11pt]{article}

\usepackage{hyperref}

\title {{Requirements}}
\author {Lucie-Aim\'{e}e Kaffee}
\date{}

\begin {document}

\maketitle

\section{deployment cycles --}
\section{non-functional requirements --}
\section{functional requirements}

We decided to have several steps of deploying the extension. 
To have a possibility to present the extension from the beginning, there was a test setup, available on \href{articleplaceholder.wmflabs.org/mediawiki}{wmflabs}. That test setup was specifically made to get a first idea of what the aims of the project are. 

The next step was to deploy the extension as a beta feature. While having certain requirements for the test set up, the beta feature was supposed to be actually used by the community and therefore needed to fulfil more requirements, building up on what was already archived in the step before.

Finally the extension would be deployed to the first wiki and therefore again needed to match other requirements again. 

\subsection{Test deploy}

This was supposed to be the most basic setup, just having a few functionalities to give an overview and learn what this extension was supposed to do. 
Those points were from a user perspective the following stories:
\begin{itemize}
\item Get to the ArticlePlaceholder via the SpecialPage URL
\item A default display of the data
\item Localisation of the whole extension
\item Show the article connected to the item, if an article on the wiki exists
\item Get to the Article Placeholders via search
\item A possibility to create an article from scratch from an Article Placeholder by giving the user the possibility to enter a title
\end{itemize}

\end {document}